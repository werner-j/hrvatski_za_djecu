\chapter{Welcome to Croatian! / Willkommen zu Kroatisch!}

\section{Introduction / Einführung}
Welcome to your first unit in Croatian! In this unit, you will learn the basics of the Croatian language, including the alphabet, pronunciation, and simple greetings.

Willkommen zu deiner ersten Einheit auf Kroatisch! In dieser Einheit lernst du die Grundlagen der kroatischen Sprache, einschließlich des Alphabets, der Aussprache und einfacher Begrüßungen.

\section{The Croatian Alphabet / Das kroatische Alphabet}

\begin{vocabulary}
The Croatian alphabet has 30 letters. It uses the Latin script (like German and English).

Das kroatische Alphabet hat 30 Buchstaben. Es verwendet das lateinische Alphabet (wie Deutsch und Englisch).

\subsection*{Complete Croatian Alphabet / Vollständiges kroatisches Alphabet}

\begin{center}
\begin{tabular}{|c|c|c|c|c|c|}
\hline
A a & B b & C c & Č č & Ć ć & D d \\
\hline
Dž dž & Đ đ & E e & F f & G g & H h \\
\hline
I i & J j & K k & L l & Lj lj & M m \\
\hline
N n & Nj nj & O o & P p & R r & S s \\
\hline
Š š & T t & U u & V v & Z z & Ž ž \\
\hline
\end{tabular}
\end{center}

\subsection*{Special Croatian Letters / Besondere kroatische Buchstaben}

Here are the special letters that don't exist in German:

Hier sind die besonderen Buchstaben, die es im Deutschen nicht gibt:

\begin{center}
\begin{tabular}{clll}
\toprule
Letter & Name & Pronunciation & Example \\
\midrule
Č, č & če & like "ch" in "church" & \textbf{č}okolada (chocolate) \\
Ć, ć & će & softer "ch" sound & ma\textbf{ć}ka (cat) \\
Dž, dž & dže & like "j" in "judge" & \textbf{dž}em (jam) \\
Đ, đ & đe & like "j" in "just" & \textbf{đ}ak (pupil) \\
Š, š & še & like "sh" in "ship" & \textbf{š}kola (school) \\
Ž, ž & že & like "s" in "measure" & \textbf{ž}ena (woman) \\
Lj, lj & elj & like "li" in "million" & \textbf{lj}ubav (love) \\
Nj, nj & enj & like "ni" in "onion" & ko\textbf{nj} (horse) \\
\bottomrule
\end{tabular}
\end{center}

\textit{Note: Audio files for pronunciation are available at [audio reference placeholder]}

\textit{Hinweis: Audiodateien zur Aussprache sind verfügbar unter [Audio-Referenz Platzhalter]}
\end{vocabulary}

\section{Pronunciation Guide / Ausspracheführer}

\begin{grammar}
\subsection*{Vowels / Vokale}

Croatian has 5 vowels, and they are always pronounced the same way:

Kroatisch hat 5 Vokale, und sie werden immer gleich ausgesprochen:

\begin{center}
\begin{tabular}{clll}
\toprule
Letter & Sound & Like in German & Example \\
\midrule
A a & "ah" & Mann & \textbf{a}nanas (pineapple) \\
E e & "eh" & Bett & \textbf{e}lefant (elephant) \\
I i & "ee" & Lied & r\textbf{i}ba (fish) \\
O o & "oh" & Boot & \textbf{o}ko (eye) \\
U u & "oo" & Fuß & \textbf{u}l (oil) \\
\bottomrule
\end{tabular}
\end{center}

\subsection*{Consonants / Konsonanten}

Most consonants sound similar to German, but pay attention to these:

Die meisten Konsonanten klingen ähnlich wie im Deutschen, aber beachte diese:

\begin{itemize}
    \item \textbf{C} - always like "ts" in "bits" (never like "k" or "s")
    \item \textbf{J} - always like "y" in "yes"
    \item \textbf{R} - rolled "r" (like in Spanish or Italian)
    \item \textbf{H} - like "h" in German "ach" (stronger than English "h")
\end{itemize}

\subsection*{Important Pronunciation Rules / Wichtige Ausspracheregeln}

\begin{enumerate}
    \item Every letter is pronounced - there are no silent letters!
    \item The stress is usually on the first syllable
    \item Double consonants are pronounced as single sounds
\end{enumerate}

\textit{Audio examples: Listen to tracks 1-10 for alphabet pronunciation}

\textit{Audiobeispiele: Höre dir die Tracks 1-10 für die Alphabet-Aussprache an}
\end{grammar}

\section{Basic Greetings / Grundlegende Grüße}

\begin{vocabulary}
\subsection*{Everyday Greetings / Alltägliche Grüße}

\begin{center}
\begin{tabular}{lll}
\toprule
Croatian & German & When to use \\
\midrule
Bok! / Bog! & Hallo! & Informal, any time \\
Dobro jutro! & Guten Morgen! & Morning (until 10am) \\
Dobar dan! & Guten Tag! & Day (10am-6pm) \\
Dobra večer! & Guten Abend! & Evening (after 6pm) \\
Laku noć! & Gute Nacht! & Before bed \\
Doviđenja! & Auf Wiedersehen! & Goodbye (formal) \\
Bok! / Ćao! & Tschüss! & Goodbye (informal) \\
\bottomrule
\end{tabular}
\end{center}

\subsection*{Polite Expressions / Höfliche Ausdrücke}

\begin{center}
\begin{tabular}{ll}
\toprule
Croatian & German \\
\midrule
Hvala! & Danke! \\
Hvala lijepo! & Vielen Dank! \\
Molim! & Bitte! \\
Molim te & Bitte (informal) \\
Molim vas & Bitte (formal) \\
Oprosti! & Entschuldigung! (informal) \\
Oprostite! & Entschuldigung! (formal) \\
Nema problema! & Kein Problem! \\
Izvini! & Verzeihung! (informal) \\
Izvinite! & Verzeihung! (formal) \\
\bottomrule
\end{tabular}
\end{center}

\subsection*{Basic Responses / Grundlegende Antworten}

\begin{center}
\begin{tabular}{ll}
\toprule
Croatian & German \\
\midrule
Da & Ja \\
Ne & Nein \\
U redu & In Ordnung / OK \\
Dobro & Gut \\
Super! & Super! \\
Odlično! & Ausgezeichnet! \\
\bottomrule
\end{tabular}
\end{center}

\textit{Audio: Listen to track 11 for greeting pronunciations}

\textit{Audio: Höre Track 11 für die Aussprache der Grüße}
\end{vocabulary}

\section{Interactive Dialogues / Interaktive Dialoge}

\begin{culture}
\subsection*{Dialogue 1: Meeting for the First Time / Dialog 1: Erstes Treffen}

\textbf{Luka:} Bok! Ja sam Luka. \\
\textit{Hi! I'm Luka. / Hallo! Ich bin Luka.}

\textbf{Ana:} Bok, Luka! Ja sam Ana. Drago mi je! \\
\textit{Hi, Luka! I'm Ana. Nice to meet you! / Hallo, Luka! Ich bin Ana. Schön, dich kennenzulernen!}

\textbf{Luka:} Drago mi je! Koliko imaš godina? \\
\textit{Nice to meet you! How old are you? / Schön, dich kennenzulernen! Wie alt bist du?}

\textbf{Ana:} Imam devet godina. A ti? \\
\textit{I'm nine years old. And you? / Ich bin neun Jahre alt. Und du?}

\textbf{Luka:} Imam deset godina. \\
\textit{I'm ten years old. / Ich bin zehn Jahre alt.}

\subsection*{Dialogue 2: In the Morning / Dialog 2: Am Morgen}

\textbf{Mama:} Dobro jutro, Ivan! \\
\textit{Good morning, Ivan! / Guten Morgen, Ivan!}

\textbf{Ivan:} Dobro jutro, mama! \\
\textit{Good morning, mom! / Guten Morgen, Mama!}

\textbf{Mama:} Kako si? \\
\textit{How are you? / Wie geht es dir?}

\textbf{Ivan:} Dobro sam, hvala! A ti? \\
\textit{I'm fine, thank you! And you? / Mir geht es gut, danke! Und dir?}

\textbf{Mama:} I ja sam dobro, hvala! \\
\textit{I'm fine too, thank you! / Mir geht es auch gut, danke!}

\subsection*{Dialogue 3: At School / Dialog 3: In der Schule}

\textbf{Učitelj:} Dobar dan, djeco! \\
\textit{Good day, children! / Guten Tag, Kinder!}

\textbf{Djeca:} Dobar dan, učitelju! \\
\textit{Good day, teacher! / Guten Tag, Herr Lehrer!}

\textbf{Učitelj:} Kako ste danas? \\
\textit{How are you today? / Wie geht es euch heute?}

\textbf{Marta:} Dobro smo, hvala! \\
\textit{We're fine, thank you! / Uns geht es gut, danke!}

\textit{Audio: Listen to tracks 12-14 for dialogue practice}

\textit{Audio: Höre Tracks 12-14 für Dialog-Übungen}
\end{culture}

\section{Introducing Yourself / Sich vorstellen}

\begin{grammar}
To introduce yourself in Croatian:

Um dich auf Kroatisch vorzustellen:

\begin{itemize}
    \item \textbf{Ja sam [name]} - Ich bin [Name]
    \item \textbf{Zovem se [name]} - Ich heiße [Name]
    \item \textbf{Imam [age] godina} - Ich bin [Alter] Jahre alt
\end{itemize}
\end{grammar}

\textbf{Example / Beispiel:}
\begin{itemize}
    \item Bok! Ja sam Ana. Imam devet godina.
    \item Hallo! Ich bin Ana. Ich bin neun Jahre alt.
\end{itemize}

\section{Numbers 1-20 / Zahlen 1-20}

\begin{vocabulary}
Learning numbers is fun and useful! Practice counting everything around you.

Zahlen lernen macht Spaß und ist nützlich! Übe, alles um dich herum zu zählen.

\begin{center}
\begin{tabular}{llll}
\toprule
\multicolumn{2}{c}{Croatian} & \multicolumn{2}{c}{German} \\
\midrule
1 - jedan & 11 - jedanaest & 1 - eins & 11 - elf \\
2 - dva & 12 - dvanaest & 2 - zwei & 12 - zwölf \\
3 - tri & 13 - trinaest & 3 - drei & 13 - dreizehn \\
4 - četiri & 14 - četrnaest & 4 - vier & 14 - vierzehn \\
5 - pet & 15 - petnaest & 5 - fünf & 15 - fünfzehn \\
6 - šest & 16 - šesnaest & 6 - sechs & 16 - sechzehn \\
7 - sedam & 17 - sedamnaest & 7 - sieben & 17 - siebzehn \\
8 - osam & 18 - osamnaest & 8 - acht & 18 - achtzehn \\
9 - devet & 19 - devetnaest & 9 - neun & 19 - neunzehn \\
10 - deset & 20 - dvadeset & 10 - zehn & 20 - zwanzig \\
\bottomrule
\end{tabular}
\end{center}

\subsection*{Number Patterns / Zahlenmuster}

Notice the pattern for 11-19:

Beachte das Muster für 11-19:

\begin{itemize}
    \item Numbers 11-19 are formed by adding \textbf{-naest} to the base number
    \item 11: jedan + naest = jedanaest
    \item 12: dva + naest = dvanaest
    \item 13: tri + naest = trinaest
    \item etc.
\end{itemize}

\textit{Audio: Listen to track 15 for number pronunciation}

\textit{Audio: Höre Track 15 für die Aussprache der Zahlen}
\end{vocabulary}

\section{Number Practice Activities / Zahlen-Übungsaktivitäten}

\begin{exercise}
\subsection*{Activity 1: Counting Game / Aktivität 1: Zählspiel}

Count these objects and write the Croatian number:

Zähle diese Gegenstände und schreibe die kroatische Zahl:

\begin{enumerate}
    \item 5 stars (stjerne) = \underline{\hspace{3cm}} (pet)
    \item 7 flowers (cvijeće) = \underline{\hspace{3cm}}
    \item 11 apples (jabuke) = \underline{\hspace{3cm}}
    \item 15 books (knjige) = \underline{\hspace{3cm}}
    \item 20 cats (mačke) = \underline{\hspace{3cm}}
\end{enumerate}

\subsection*{Activity 2: Math in Croatian / Aktivität 2: Mathematik auf Kroatisch}

Solve these simple math problems and write the answer in Croatian:

Löse diese einfachen Matheaufgaben und schreibe die Antwort auf Kroatisch:

\begin{enumerate}
    \item pet + tri = \underline{\hspace{3cm}}
    \item deset - dva = \underline{\hspace{3cm}}
    \item sedam + šest = \underline{\hspace{3cm}}
    \item petnaest - pet = \underline{\hspace{3cm}}
    \item devet + jedan = \underline{\hspace{3cm}}
\end{enumerate}

\textit{Tip: Plus = plus, Minus = minus}

\subsection*{Activity 3: Telephone Numbers / Aktivität 3: Telefonnummern}

In Croatian, we say telephone numbers digit by digit. Practice saying these numbers:

Auf Kroatisch sagen wir Telefonnummern Ziffer für Ziffer. Übe, diese Zahlen zu sagen:

\begin{enumerate}
    \item 385-1-234-5678: tri-osam-pet, jedan, dva-tri-četiri, pet-šest-sedam-osam
    \item Your phone number: \underline{\hspace{8cm}}
    \item Your age: Imam \underline{\hspace{3cm}} godina.
\end{enumerate}
\end{exercise}

\section{Croatia - Basic Facts / Kroatien - Grundlegende Fakten}

\begin{culture}
\textbf{Welcome to Croatia! / Willkommen in Kroatien!}

Croatia (Hrvatska in Croatian) is a beautiful country in Southeast Europe. Let's learn about it!

Kroatien (Hrvatska auf Kroatisch) ist ein wunderschönes Land in Südosteuropa. Lass uns darüber lernen!

\subsection*{Quick Facts / Schnelle Fakten}

\begin{itemize}
    \item \textbf{Official Name / Offizieller Name:} Republika Hrvatska (Republic of Croatia)
    \item \textbf{Capital / Hauptstadt:} Zagreb (population: about 800,000)
    \item \textbf{Population / Bevölkerung:} About 4 million people / Etwa 4 Millionen Menschen
    \item \textbf{Language / Sprache:} Croatian (Hrvatski) / Kroatisch
    \item \textbf{Currency / Währung:} Euro (€) since 2023 / seit 2023
    \item \textbf{Location / Lage:} Southeast Europe, Adriatic Sea coast / Südosteuropa, Adriaküste
    \item \textbf{EU Member / EU-Mitglied:} Since July 1, 2013 / Seit 1. Juli 2013
    \item \textbf{Neighbors / Nachbarländer:} Slovenia, Hungary, Serbia, Bosnia and Herzegovina, Montenegro
\end{itemize}

\subsection*{The Croatian Flag / Die kroatische Flagge}

The Croatian flag has three horizontal stripes:
\begin{itemize}
    \item \textcolor{croatianred}{\textbf{Red}} (top) - represents the blood of Croatian heroes
    \item \textbf{White} (middle) - represents peace
    \item \textcolor{croatianblue}{\textbf{Blue}} (bottom) - represents the sky and sea
\end{itemize}

In the center is the Croatian coat of arms with a red and white checkerboard pattern (šahovnica).

Die kroatische Flagge hat drei horizontale Streifen:
\begin{itemize}
    \item \textcolor{croatianred}{\textbf{Rot}} (oben) - steht für das Blut kroatischer Helden
    \item \textbf{Weiß} (Mitte) - steht für Frieden
    \item \textcolor{croatianblue}{\textbf{Blau}} (unten) - steht für Himmel und Meer
\end{itemize}

\subsection*{Geography / Geographie}

Croatia has three main regions:

Kroatien hat drei Hauptregionen:

\begin{enumerate}
    \item \textbf{Coastal Region (Jadranska regija):} 
    \begin{itemize}
        \item Beautiful beaches along the Adriatic Sea / Wunderschöne Strände an der Adria
        \item Over 1,000 islands! / Über 1.000 Inseln!
        \item Famous cities: Split, Dubrovnik, Zadar, Rijeka
    \end{itemize}
    
    \item \textbf{Central Croatia (Središnja Hrvatska):}
    \begin{itemize}
        \item The capital Zagreb is here / Die Hauptstadt Zagreb liegt hier
        \item Rolling hills and agriculture / Hügelige Landschaft und Landwirtschaft
    \end{itemize}
    
    \item \textbf{Eastern Croatia (Slavonija):}
    \begin{itemize}
        \item Flat plains / Flache Ebenen
        \item Important for farming / Wichtig für die Landwirtschaft
    \end{itemize}
\end{enumerate}

\subsection*{Major Cities / Große Städte}

\begin{itemize}
    \item \textbf{Zagreb} - The capital, largest city
    \item \textbf{Split} - Second largest, ancient Roman palace
    \item \textbf{Dubrovnik} - "Pearl of the Adriatic," medieval walls
    \item \textbf{Rijeka} - Important port city
    \item \textbf{Zadar} - Historic city with Roman ruins
    \item \textbf{Pula} - Famous Roman amphitheater
\end{itemize}

\textit{Photos: See illustrations of Croatian landscapes and cities}

\textit{Fotos: Siehe Illustrationen der kroatischen Landschaften und Städte}
\end{culture}

\section{Fun Facts About Croatia / Spaßige Fakten über Kroatien}

\begin{culture}
\subsection*{Did You Know? / Wusstest du das?}

\begin{enumerate}
    \item \textbf{The Necktie (Krawatte):} The necktie was invented by Croatian soldiers in the 17th century! The word "cravat" comes from "Croat."
    
    \textit{Die Krawatte wurde von kroatischen Soldaten im 17. Jahrhundert erfunden! Das Wort "Krawatte" kommt von "Kroate."}
    
    \item \textbf{Dalmatian Dogs:} These spotted dogs originally came from Dalmatia, a region in Croatia!
    
    \textit{Diese gepunkteten Hunde kamen ursprünglich aus Dalmatien, einer Region in Kroatien!}
    
    \item \textbf{Beautiful Nature:} Croatia has 8 National Parks and 11 Nature Parks, including the famous Plitvice Lakes.
    
    \textit{Kroatien hat 8 Nationalparks und 11 Naturparks, einschließlich der berühmten Plitvicer Seen.}
    
    \item \textbf{Island Paradise:} Croatia has 1,244 islands, but only 48 are inhabited!
    
    \textit{Kroatien hat 1.244 Inseln, aber nur 48 sind bewohnt!}
    
    \item \textbf{Ancient History:} The Roman Emperor Diocletian built his retirement palace in Split, which is now a UNESCO World Heritage site!
    
    \textit{Der römische Kaiser Diokletian baute seinen Altersruhesitz in Split, der jetzt UNESCO-Weltkulturerbe ist!}
    
    \item \textbf{Game of Thrones:} The popular TV series was filmed in Dubrovnik! The city became "King's Landing."
    
    \textit{Die beliebte TV-Serie wurde in Dubrovnik gedreht! Die Stadt wurde zu "Königsmund."}
    
    \item \textbf{Nikola Tesla:} The famous inventor was born in Croatia (though he later lived in America).
    
    \textit{Der berühmte Erfinder wurde in Kroatien geboren (obwohl er später in Amerika lebte).}
    
    \item \textbf{Beautiful Coast:} The Adriatic Sea along Croatia's coast is crystal clear and perfect for swimming!
    
    \textit{Das Adriatische Meer entlang der kroatischen Küste ist kristallklar und perfekt zum Schwimmen!}
    
    \item \textbf{Music and Dance:} Croatia has many traditional folk dances and a unique musical instrument called the tamburica.
    
    \textit{Kroatien hat viele traditionelle Volkstänze und ein einzigartiges Musikinstrument namens Tamburica.}
    
    \item \textbf{Sports Champions:} Croatia is famous for water polo, handball, and football (soccer). The national football team won 2nd place in the 2018 World Cup!
    
    \textit{Kroatien ist berühmt für Wasserball, Handball und Fußball. Die Nationalmannschaft wurde 2018 Vizeweltmeister!}
\end{enumerate}

\subsection*{Croatian Symbols / Kroatische Symbole}

\begin{itemize}
    \item \textbf{National Animal:} Pine marten (kuna) - also the name of the old Croatian currency!
    \item \textbf{National Flower:} Iris croatica (Croatian iris)
    \item \textbf{National Tree:} Oak tree (hrast)
\end{itemize}

\textit{Illustrations: Pictures of Croatian symbols and landmarks}

\textit{Illustrationen: Bilder der kroatischen Symbole und Wahrzeichen}
\end{culture}

\section{Exercises / Übungen}

\begin{exercise}
\textbf{Exercise 1: Alphabet Practice / Übung 1: Alphabet-Übung}

A) Write the complete Croatian alphabet (all 30 letters):

Schreibe das vollständige kroatische Alphabet (alle 30 Buchstaben):

\vspace{2cm}

B) Circle the 8 special letters that don't exist in German:

Kreise die 8 besonderen Buchstaben ein, die es im Deutschen nicht gibt:

\vspace{1cm}

C) Match the Croatian words with their pronunciations:

Verbinde die kroatischen Wörter mit ihrer Aussprache:

\begin{tabular}{ll}
1. škola & a) "chocolate" \\
2. čokolada & b) "school" \\
3. mačka & c) "jam" \\
4. džem & d) "cat" \\
\end{tabular}

\vspace{2cm}

\textbf{Exercise 2: Greetings / Übung 2: Grüße}

A) Fill in the blanks with the correct Croatian greeting:

Fülle die Lücken mit dem richtigen kroatischen Gruß:

\begin{enumerate}
    \item At 8:00 am you say: \underline{\hspace{5cm}}
    \item At 2:00 pm you say: \underline{\hspace{5cm}}
    \item At 8:00 pm you say: \underline{\hspace{5cm}}
    \item Before going to bed: \underline{\hspace{5cm}}
    \item When saying goodbye: \underline{\hspace{5cm}}
\end{enumerate}

B) Translate these polite expressions into Croatian:

Übersetze diese höflichen Ausdrücke ins Kroatische:

\begin{enumerate}
    \item Thank you: \underline{\hspace{5cm}}
    \item Please: \underline{\hspace{5cm}}
    \item Excuse me: \underline{\hspace{5cm}}
    \item No problem: \underline{\hspace{5cm}}
\end{enumerate}

\vspace{2cm}

\textbf{Exercise 3: Introduction / Übung 3: Vorstellung}

Write a short introduction about yourself in Croatian (at least 5 sentences). Include:
- Your name
- Your age
- A greeting
- How you are feeling
- A goodbye

Schreibe eine kurze Vorstellung über dich auf Kroatisch (mindestens 5 Sätze). Füge hinzu:
- Deinen Namen
- Dein Alter
- Einen Gruß
- Wie es dir geht
- Einen Abschied

\vspace{5cm}

\textbf{Exercise 4: Dialogue Practice / Übung 4: Dialog-Übung}

Complete this dialogue with appropriate Croatian phrases:

Vervollständige diesen Dialog mit passenden kroatischen Sätzen:

\textbf{Marko:} Bok! Ja sam Marko. \\
\textbf{You:} \underline{\hspace{8cm}} \\
\textbf{Marko:} Drago mi je! Koliko imaš godina? \\
\textbf{You:} \underline{\hspace{8cm}} \\
\textbf{Marko:} Kako si danas? \\
\textbf{You:} \underline{\hspace{8cm}} \\
\textbf{Marko:} I ja sam dobro! Doviđenja! \\
\textbf{You:} \underline{\hspace{8cm}}

\vspace{2cm}

\textbf{Exercise 5: Numbers / Übung 5: Zahlen}

A) Write these numbers in Croatian:

Schreibe diese Zahlen auf Kroatisch:

\begin{enumerate}
    \item 7: \underline{\hspace{4cm}}
    \item 13: \underline{\hspace{4cm}}
    \item 18: \underline{\hspace{4cm}}
    \item 5: \underline{\hspace{4cm}}
    \item 20: \underline{\hspace{4cm}}
\end{enumerate}

B) Practice counting from 1 to 20 in Croatian. Then answer these questions in Croatian:

Übe das Zählen von 1 bis 20 auf Kroatisch. Beantworte dann diese Fragen auf Kroatisch:

\begin{itemize}
    \item How old are you? \textit{Koliko imaš godina?} \\
    Answer: \underline{\hspace{8cm}}
    
    \item Count your fingers (1-10): \\
    Answer: \underline{\hspace{8cm}}
    
    \item How many students are in your class? \\
    Answer: U mojoj školi ima \underline{\hspace{4cm}} učenika.
\end{itemize}

\vspace{2cm}

\textbf{Exercise 6: Croatia Facts / Übung 6: Kroatien-Fakten}

Answer these questions about Croatia:

Beantworte diese Fragen über Kroatien:

\begin{enumerate}
    \item What is the capital of Croatia? \underline{\hspace{6cm}}
    \item How many islands does Croatia have? \underline{\hspace{6cm}}
    \item What are the colors of the Croatian flag? \underline{\hspace{6cm}}
    \item Name two famous Croatian cities: \underline{\hspace{6cm}}
    \item What animal is the national symbol? \underline{\hspace{6cm}}
    \item Who invented the necktie? \underline{\hspace{6cm}}
\end{enumerate}

\vspace{2cm}

\textbf{Exercise 7: Creative Activity / Übung 7: Kreative Aktivität}

A) Draw the Croatian flag and label the colors in Croatian.

Male die kroatische Flagge und beschrifte die Farben auf Kroatisch.

\vspace{5cm}

B) Create a mini-poster about Croatia with 5 fun facts you learned. Use both Croatian and German.

Erstelle ein Mini-Poster über Kroatien mit 5 lustigen Fakten, die du gelernt hast. Verwende Kroatisch und Deutsch.

\vspace{5cm}

\textbf{Exercise 8: Listening Practice / Übung 8: Hörübung}

\textit{Listen to audio tracks 1-15 and complete the following:}

\textit{Höre dir die Audiotracks 1-15 an und vervollständige Folgendes:}

\begin{enumerate}
    \item Listen to track 11 and write down all the greetings you hear:
    
    \vspace{2cm}
    
    \item Listen to track 15 and write the numbers in Croatian as you hear them:
    
    \vspace{2cm}
    
    \item Listen to dialogues 12-14 and answer: What are the names of the people in the dialogues?
    
    \vspace{2cm}
\end{enumerate}
\end{exercise}

\section{Summary / Zusammenfassung}

In this unit, you learned:
\begin{itemize}
    \item The Croatian alphabet with 30 letters (including complete alphabet table)
    \item Special Croatian letters: č, ć, dž, đ, š, ž, lj, nj (all 8 special letters)
    \item Detailed pronunciation guide for vowels and consonants
    \item Basic greetings and polite expressions for different times of day
    \item How to introduce yourself and have simple conversations
    \item Interactive dialogues in Croatian with translations
    \item Numbers 1-20 with number patterns and practice activities
    \item Comprehensive facts about Croatia: geography, flag, cities
    \item 10 fun facts about Croatia including culture, history, and sports
    \item Croatian national symbols (animal, flower, tree)
\end{itemize}

In dieser Einheit hast du gelernt:
\begin{itemize}
    \item Das kroatische Alphabet mit 30 Buchstaben (einschließlich vollständiger Alphabettabelle)
    \item Besondere kroatische Buchstaben: č, ć, dž, đ, š, ž, lj, nj (alle 8 besonderen Buchstaben)
    \item Ausführlicher Ausspracheführer für Vokale und Konsonanten
    \item Grundlegende Begrüßungen und höfliche Ausdrücke für verschiedene Tageszeiten
    \item Wie man sich vorstellt und einfache Gespräche führt
    \item Interaktive Dialoge auf Kroatisch mit Übersetzungen
    \item Zahlen 1-20 mit Zahlenmustern und Übungsaktivitäten
    \item Umfassende Fakten über Kroatien: Geographie, Flagge, Städte
    \item 10 lustige Fakten über Kroatien einschließlich Kultur, Geschichte und Sport
    \item Kroatische nationale Symbole (Tier, Blume, Baum)
\end{itemize}

\vspace{1cm}

\textbf{Great job completing Unit 1! You're now ready to move on to Unit 2!}

\textbf{Gut gemacht, Einheit 1 abgeschlossen! Du bist jetzt bereit für Einheit 2!}

\textit{Audio reminder: Review tracks 1-15 to reinforce your learning}

\textit{Audio-Erinnerung: Überprüfe die Tracks 1-15, um dein Lernen zu festigen}
