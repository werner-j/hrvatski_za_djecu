\chapter{Willkommen zu Kroatisch!}

\section{Einführung}
Willkommen zu deiner ersten Einheit auf Kroatisch! In dieser Einheit lernst du die Grundlagen der kroatischen Sprache, einschließlich des Alphabets, der Aussprache und einfacher Begrüßungen.

\section{Das kroatische Alphabet}

\begin{vocabulary}
Das kroatische Alphabet hat 30 Buchstaben. Es verwendet das lateinische Alphabet (wie Deutsch).

\subsection*{Vollständiges kroatisches Alphabet}

\begin{center}
\begin{tabular}{|c|c|c|c|c|c|}
\hline
A a & B b & C c & Č č & Ć ć & D d \\
\hline
Dž dž & Đ đ & E e & F f & G g & H h \\
\hline
I i & J j & K k & L l & Lj lj & M m \\
\hline
N n & Nj nj & O o & P p & R r & S s \\
\hline
Š š & T t & U u & V v & Z z & Ž ž \\
\hline
\end{tabular}
\end{center}

\subsection*{Besondere kroatische Buchstaben}

Hier sind die besonderen Buchstaben, die es im Deutschen nicht gibt:

\begin{center}
\begin{tabular}{clll}
\toprule
Buchstabe & Name & Aussprache & Beispiel \\
\midrule
Č, č & če & wie \glqq tsch\grqq{} in \glqq Tschüss\grqq{} & \textbf{č}okolada (Schokolade) \\
Ć, ć & će & weicherer \glqq tsch\grqq{}-Laut & ma\textbf{ć}ka (Katze) \\
Dž, dž & dže & wie \glqq dsch\grqq{} in \glqq Dschungel\grqq{} & \textbf{dž}em (Marmelade) \\
Đ, đ & đe & weicherer \glqq dsch\grqq{}-Laut & \textbf{đ}ak (Schüler) \\
Š, š & še & wie \glqq sch\grqq{} in \glqq Schule\grqq{} & \textbf{š}kola (Schule) \\
Ž, ž & že & stimmhaftes \glqq sch\grqq{} & \textbf{ž}ena (Frau) \\
Lj, lj & elj & palatalisiertes \glqq l\grqq{} & \textbf{lj}ubav (Liebe) \\
Nj, nj & enj & palatalisiertes \glqq n\grqq{} & ko\textbf{nj} (Pferd) \\
\bottomrule
\end{tabular}
\end{center}

\textit{Note: Audio files for pronunciation are available at [audio reference placeholder]}

\textit{Hinweis: Audiodateien zur Aussprache sind verfügbar unter [Audio-Referenz Platzhalter]}
\end{vocabulary}

\section{Ausspracheführer}

\begin{grammar}
\subsection*{Vokale}

Kroatisch hat 5 Vokale, und sie werden immer gleich ausgesprochen:

\begin{center}
\begin{tabular}{clll}
\toprule
Buchstabe & Laut & Wie im Deutschen & Beispiel \\
\midrule
A a & \glqq ah\grqq{} & Mann & \textbf{a}nanas (Ananas) \\
E e & \glqq eh\grqq{} & Bett & \textbf{e}lefant (Elefant) \\
I i & \glqq ee\grqq{} & Lied & r\textbf{i}ba (Fisch) \\
O o & \glqq oh\grqq{} & Boot & \textbf{o}ko (Auge) \\
U u & \glqq oo\grqq{} & Fuß & \textbf{u}l (Öl) \\
\bottomrule
\end{tabular}
\end{center}

\subsection*{Konsonanten}

Die meisten Konsonanten klingen ähnlich wie im Deutschen, aber beachte diese:

\begin{itemize}
    \item \textbf{C} - immer wie \glqq ts\grqq{} in \glqq Zeit\grqq{} (niemals wie \glqq k\grqq{} oder \glqq s\grqq{})
    \item \textbf{J} - immer wie \glqq j\grqq{} in \glqq ja\grqq{}
    \item \textbf{R} - gerolltes \glqq r\grqq{} (wie im Spanischen oder Italienischen)
    \item \textbf{H} - wie deutsches \glqq ch\grqq{} in \glqq ach\grqq{} (NICHT wie das \glqq h\grqq{} in \glqq Haus\grqq{})
\end{itemize}

\subsection*{Wichtige Ausspracheregeln}

\begin{enumerate}
    \item Jeder Buchstabe wird ausgesprochen - es gibt keine stummen Buchstaben!
    \item Die Betonung liegt normalerweise auf der ersten Silbe
    \item Doppelkonsonanten werden als Einzellaute ausgesprochen
\end{enumerate}

\textit{Audio examples: Listen to tracks 1-10 for alphabet pronunciation}

\textit{Audiobeispiele: Höre dir die Tracks 1-10 für die Alphabet-Aussprache an}
\end{grammar}

\section{Grundlegende Grüße}

\begin{vocabulary}
\subsection*{Alltägliche Grüße}

\begin{center}
\begin{tabular}{lll}
\toprule
Kroatisch & Deutsch & Wann zu verwenden \\
\midrule
Bok! / Bog! & Hallo! & Informell, jederzeit \\
Dobro jutro! & Guten Morgen! & Morgen (bis 10 Uhr) \\
Dobar dan! & Guten Tag! & Tag (10-18 Uhr) \\
Dobra večer! & Guten Abend! & Abend (nach 18 Uhr) \\
Laku noć! & Gute Nacht! & Vor dem Schlafengehen \\
Doviđenja! & Auf Wiedersehen! & Abschied (formell) \\
Bok! / Ćao! & Tschüss! & Abschied (informell) \\
\bottomrule
\end{tabular}
\end{center}

\subsection*{Höfliche Ausdrücke}

\begin{center}
\begin{tabular}{ll}
\toprule
Kroatisch & Deutsch \\
\midrule
Hvala! & Danke! \\
Hvala lijepo! & Vielen Dank! \\
Molim! & Bitte! \\
Molim te & Bitte (informell) \\
Molim vas & Bitte (formell) \\
Oprosti! & Entschuldigung! (informell) \\
Oprostite! & Entschuldigung! (formell) \\
Nema problema! & Kein Problem! \\
Izvini! & Verzeihung! (informell) \\
Izvinite! & Verzeihung! (formell) \\
\bottomrule
\end{tabular}
\end{center}

\subsection*{Grundlegende Antworten}

\begin{center}
\begin{tabular}{ll}
\toprule
Kroatisch & Deutsch \\
\midrule
Da & Ja \\
Ne & Nein \\
U redu & In Ordnung / OK \\
Dobro & Gut \\
Super! & Super! \\
Odlično! & Ausgezeichnet! \\
\bottomrule
\end{tabular}
\end{center}

\textit{Audio: Listen to track 11 for greeting pronunciations}

\textit{Audio: Höre Track 11 für die Aussprache der Grüße}
\end{vocabulary}

\section{Interaktive Dialoge}

\begin{culture}
\subsection*{Dialog 1: Erstes Treffen}

\textbf{Luka:} Bok! Ja sam Luka. \\
\textit{Hallo! Ich bin Luka.}

\textbf{Ana:} Bok, Luka! Ja sam Ana. Drago mi je! \\
\textit{Hallo, Luka! Ich bin Ana. Schön, dich kennenzulernen!}

\textbf{Luka:} Drago mi je! Koliko imaš godina? \\
\textit{Schön, dich kennenzulernen! Wie alt bist du?}

\textbf{Ana:} Imam devet godina. A ti? \\
\textit{Ich bin neun Jahre alt. Und du?}

\textbf{Luka:} Imam deset godina. \\
\textit{Ich bin zehn Jahre alt.}

\subsection*{Dialog 2: Am Morgen}

\textbf{Mama:} Dobro jutro, Ivan! \\
\textit{Guten Morgen, Ivan!}

\textbf{Ivan:} Dobro jutro, mama! \\
\textit{Guten Morgen, Mama!}

\textbf{Mama:} Kako si? \\
\textit{Wie geht es dir?}

\textbf{Ivan:} Dobro sam, hvala! A ti? \\
\textit{Mir geht es gut, danke! Und dir?}

\textbf{Mama:} I ja sam dobro, hvala! \\
\textit{Mir geht es auch gut, danke!}

\subsection*{Dialog 3: In der Schule}

\textbf{Učitelj:} Dobar dan, djeco! \\
\textit{Guten Tag, Kinder!}

\textbf{Djeca:} Dobar dan, učitelju! \\
\textit{Guten Tag, Herr Lehrer!}

\textbf{Učitelj:} Kako ste danas? \\
\textit{Wie geht es euch heute?}

\textbf{Marta:} Dobro smo, hvala! \\
\textit{Uns geht es gut, danke!}

\textit{Audio: Listen to tracks 12-14 for dialogue practice}

\textit{Audio: Höre Tracks 12-14 für Dialog-Übungen}
\end{culture}

\section{Sich vorstellen}

\begin{grammar}
Um dich auf Kroatisch vorzustellen:

\begin{itemize}
    \item \textbf{Ja sam [name]} - Ich bin [Name]
    \item \textbf{Zovem se [name]} - Ich heiße [Name]
    \item \textbf{Imam [age] godina} - Ich bin [Alter] Jahre alt
\end{itemize}
\end{grammar}

\textbf{Beispiel:}
\begin{itemize}
    \item Bok! Ja sam Ana. Imam devet godina.
    \item Hallo! Ich bin Ana. Ich bin neun Jahre alt.
\end{itemize}

\section{Zahlen 1-20}

\begin{vocabulary}
Zahlen lernen macht Spaß und ist nützlich! Übe, alles um dich herum zu zählen.

\begin{center}
\begin{tabular}{llll}
\toprule
\multicolumn{2}{c}{Kroatisch} & \multicolumn{2}{c}{Deutsch} \\
\midrule
1 - jedan & 11 - jedanaest & 1 - eins & 11 - elf \\
2 - dva & 12 - dvanaest & 2 - zwei & 12 - zwölf \\
3 - tri & 13 - trinaest & 3 - drei & 13 - dreizehn \\
4 - četiri & 14 - četrnaest & 4 - vier & 14 - vierzehn \\
5 - pet & 15 - petnaest & 5 - fünf & 15 - fünfzehn \\
6 - šest & 16 - šesnaest & 6 - sechs & 16 - sechzehn \\
7 - sedam & 17 - sedamnaest & 7 - sieben & 17 - siebzehn \\
8 - osam & 18 - osamnaest & 8 - acht & 18 - achtzehn \\
9 - devet & 19 - devetnaest & 9 - neun & 19 - neunzehn \\
10 - deset & 20 - dvadeset & 10 - zehn & 20 - zwanzig \\
\bottomrule
\end{tabular}
\end{center}

\subsection*{Zahlenmuster}

Beachte das Muster für 11-19:

\begin{itemize}
    \item Die Zahlen 11-19 werden gebildet, indem \textbf{-naest} an die Basiszahl angehängt wird
    \item 11: jedan + naest = jedanaest
    \item 12: dva + naest = dvanaest
    \item 13: tri + naest = trinaest
    \item usw.
\end{itemize}

\textit{Audio: Listen to track 15 for number pronunciation}

\textit{Audio: Höre Track 15 für die Aussprache der Zahlen}
\end{vocabulary}

\section{Zahlen-Übungsaktivitäten}

\begin{exercise}
\subsection*{Aktivität 1: Zählspiel}

Zähle diese Gegenstände und schreibe die kroatische Zahl:

\begin{enumerate}
    \item 5 Sterne (stjerne) = \underline{\hspace{3cm}} (pet)
    \item 7 Blumen (cvijeće) = \underline{\hspace{3cm}}
    \item 11 Äpfel (jabuke) = \underline{\hspace{3cm}}
    \item 15 Bücher (knjige) = \underline{\hspace{3cm}}
    \item 20 Katzen (mačke) = \underline{\hspace{3cm}}
\end{enumerate}

\subsection*{Aktivität 2: Mathematik auf Kroatisch}

Löse diese einfachen Matheaufgaben und schreibe die Antwort auf Kroatisch:

\begin{enumerate}
    \item pet + tri = \underline{\hspace{3cm}}
    \item deset - dva = \underline{\hspace{3cm}}
    \item sedam + šest = \underline{\hspace{3cm}}
    \item petnaest - pet = \underline{\hspace{3cm}}
    \item devet + jedan = \underline{\hspace{3cm}}
\end{enumerate}

\textit{Tipp: Plus = plus, Minus = minus}

\subsection*{Aktivität 3: Telefonnummern}

Auf Kroatisch sagen wir Telefonnummern Ziffer für Ziffer. Übe, diese Zahlen zu sagen:

\begin{enumerate}
    \item 385-1-234-5678: tri-osam-pet, jedan, dva-tri-četiri, pet-šest-sedam-osam
    \item Deine Telefonnummer: \underline{\hspace{8cm}}
    \item Dein Alter: Imam \underline{\hspace{3cm}} godina.
\end{enumerate}
\end{exercise}

\section{Kroatien - Grundlegende Fakten}

\begin{culture}
\textbf{Willkommen in Kroatien!}

Kroatien (Hrvatska auf Kroatisch) ist ein wunderschönes Land in Südosteuropa. Lass uns darüber lernen!

\subsection*{Schnelle Fakten}

\begin{itemize}
    \item \textbf{Offizieller Name:} Republika Hrvatska (Republik Kroatien)
    \item \textbf{Hauptstadt:} Zagreb (Bevölkerung: ca. 800.000)
    \item \textbf{Bevölkerung:} Etwa 4 Millionen Menschen
    \item \textbf{Sprache:} Hrvatski (Kroatisch)
    \item \textbf{Währung:} Euro (€) seit 2023
    \item \textbf{Lage:} Südosteuropa, Adriaküste
    \item \textbf{EU-Mitglied:} Seit 1. Juli 2013
    \item \textbf{Nachbarländer:} Slowenien, Ungarn, Serbien, Bosnien und Herzegowina, Montenegro
\end{itemize}

\subsection*{Die kroatische Flagge}

Die kroatische Flagge hat drei horizontale Streifen:
\begin{itemize}
    \item \textcolor{croatianred}{\textbf{Rot}} (oben) - steht für das Blut kroatischer Helden
    \item \textbf{Weiß} (Mitte) - steht für Frieden
    \item \textcolor{croatianblue}{\textbf{Blau}} (unten) - steht für Himmel und Meer
\end{itemize}

In der Mitte befindet sich das kroatische Wappen mit einem rot-weißen Schachbrettmuster (šahovnica).

\subsection*{Geographie}

Kroatien hat drei Hauptregionen:

\begin{enumerate}
    \item \textbf{Küstenregion (Jadranska regija):} 
    \begin{itemize}
        \item Wunderschöne Strände an der Adria
        \item Über 1.000 Inseln!
        \item Berühmte Städte: Split, Dubrovnik, Zadar, Rijeka
    \end{itemize}
    
    \item \textbf{Zentralkroatien (Središnja Hrvatska):}
    \begin{itemize}
        \item Die Hauptstadt Zagreb liegt hier
        \item Hügelige Landschaft und Landwirtschaft
    \end{itemize}
    
    \item \textbf{Ostkroatien (Slavonija):}
    \begin{itemize}
        \item Flache Ebenen
        \item Wichtig für die Landwirtschaft
    \end{itemize}
\end{enumerate}

\subsection*{Große Städte}

\begin{itemize}
    \item \textbf{Zagreb} - Die Hauptstadt, größte Stadt
    \item \textbf{Split} - Zweitgrößte Stadt, antiker römischer Palast
    \item \textbf{Dubrovnik} - \glqq Perle der Adria\grqq{}, mittelalterliche Mauern
    \item \textbf{Rijeka} - Wichtige Hafenstadt
    \item \textbf{Zadar} - Historische Stadt mit römischen Ruinen
    \item \textbf{Pula} - Berühmtes römisches Amphitheater
\end{itemize}

\textit{Fotos: Siehe Illustrationen der kroatischen Landschaften und Städte}
\end{culture}

\section{Spaßige Fakten über Kroatien}

\begin{culture}
\subsection*{Wusstest du das?}

\begin{enumerate}
    \item \textbf{Die Krawatte:} Die Krawatte wurde von kroatischen Soldaten im 17. Jahrhundert erfunden! Das Wort \glqq Krawatte\grqq{} kommt von \glqq Kroate.\grqq{}
    
    \item \textbf{Dalmatiner-Hunde:} Diese gepunkteten Hunde kamen ursprünglich aus Dalmatien, einer Region in Kroatien!
    
    \item \textbf{Wunderschöne Natur:} Kroatien hat 8 Nationalparks und 11 Naturparks, einschließlich der berühmten Plitvicer Seen.
    
    \item \textbf{Inselparadies:} Kroatien hat 1.244 Inseln, aber nur 48 sind bewohnt!
    
    \item \textbf{Antike Geschichte:} Der römische Kaiser Diokletian baute seinen Altersruhesitz in Split, der jetzt UNESCO-Weltkulturerbe ist!
    
    \item \textbf{Game of Thrones:} Die beliebte TV-Serie wurde in Dubrovnik gedreht! Die Stadt wurde zu \glqq Königsmund.\grqq{}
    
    \item \textbf{Nikola Tesla:} Der berühmte Erfinder wurde in Kroatien geboren (obwohl er später in Amerika lebte).
    
    \item \textbf{Wunderschöne Küste:} Das Adriatische Meer entlang der kroatischen Küste ist kristallklar und perfekt zum Schwimmen!
    
    \item \textbf{Musik und Tanz:} Kroatien hat viele traditionelle Volkstänze und ein einzigartiges Musikinstrument namens Tamburica.
    
    \item \textbf{Sport-Champions:} Kroatien ist berühmt für Wasserball, Handball und Fußball. Die Nationalmannschaft wurde 2018 Vizeweltmeister!
\end{enumerate}

\subsection*{Kroatische Symbole}

\begin{itemize}
    \item \textbf{Nationaltier:} Baummarder (kuna) - auch der Name der alten kroatischen Währung!
    \item \textbf{Nationalblume:} Iris croatica (Kroatische Iris)
    \item \textbf{Nationalbaum:} Eiche (hrast)
\end{itemize}

\textit{Illustrationen: Bilder der kroatischen Symbole und Wahrzeichen}
\end{culture}

\section{Übungen}

\begin{exercise}
\textbf{Übung 1: Alphabet-Übung}

A) Schreibe das vollständige kroatische Alphabet (alle 30 Buchstaben):

\vspace{2cm}

B) Kreise die 8 besonderen Buchstaben ein, die es im Deutschen nicht gibt:

\vspace{1cm}

C) Verbinde die kroatischen Wörter mit ihrer Aussprache:

\begin{tabular}{ll}
1. škola & a) \glqq Schokolade\grqq{} \\
2. čokolada & b) \glqq Schule\grqq{} \\
3. mačka & c) \glqq Marmelade\grqq{} \\
4. džem & d) \glqq Katze\grqq{} \\
\end{tabular}

\vspace{2cm}

\textbf{Übung 2: Grüße}

A) Fülle die Lücken mit dem richtigen kroatischen Gruß:

\begin{enumerate}
    \item Um 8:00 Uhr morgens sagst du: \underline{\hspace{5cm}}
    \item Um 14:00 Uhr sagst du: \underline{\hspace{5cm}}
    \item Um 20:00 Uhr sagst du: \underline{\hspace{5cm}}
    \item Vor dem Schlafengehen: \underline{\hspace{5cm}}
    \item Beim Abschied: \underline{\hspace{5cm}}
\end{enumerate}

B) Übersetze diese höflichen Ausdrücke ins Kroatische:

\begin{enumerate}
    \item Danke: \underline{\hspace{5cm}}
    \item Bitte: \underline{\hspace{5cm}}
    \item Entschuldigung: \underline{\hspace{5cm}}
    \item Kein Problem: \underline{\hspace{5cm}}
\end{enumerate}

\vspace{2cm}

\textbf{Übung 3: Vorstellung}

Schreibe eine kurze Vorstellung über dich auf Kroatisch (mindestens 5 Sätze). Füge hinzu:
- Deinen Namen
- Dein Alter
- Einen Gruß
- Wie es dir geht
- Einen Abschied

\vspace{5cm}

\textbf{Übung 4: Dialog-Übung}

Vervollständige diesen Dialog mit passenden kroatischen Sätzen:

\textbf{Marko:} Bok! Ja sam Marko. \\
\textbf{Du:} \underline{\hspace{8cm}} \\
\textbf{Marko:} Drago mi je! Koliko imaš godina? \\
\textbf{Du:} \underline{\hspace{8cm}} \\
\textbf{Marko:} Kako si danas? \\
\textbf{Du:} \underline{\hspace{8cm}} \\
\textbf{Marko:} I ja sam dobro! Doviđenja! \\
\textbf{Du:} \underline{\hspace{8cm}}

\vspace{2cm}

\textbf{Übung 5: Zahlen}

A) Schreibe diese Zahlen auf Kroatisch:

\begin{enumerate}
    \item 7: \underline{\hspace{4cm}}
    \item 13: \underline{\hspace{4cm}}
    \item 18: \underline{\hspace{4cm}}
    \item 5: \underline{\hspace{4cm}}
    \item 20: \underline{\hspace{4cm}}
\end{enumerate}

B) Übe das Zählen von 1 bis 20 auf Kroatisch. Beantworte dann diese Fragen auf Kroatisch:

\begin{itemize}
    \item Wie alt bist du? \textit{Koliko imaš godina?} \\
    Antwort: \underline{\hspace{8cm}}
    
    \item Zähle deine Finger (1-10): \\
    Antwort: \underline{\hspace{8cm}}
    
    \item Wie viele Schüler sind in deiner Klasse? \\
    Antwort: U mojoj školi ima \underline{\hspace{4cm}} učenika.
\end{itemize}

\vspace{2cm}

\textbf{Übung 6: Kroatien-Fakten}

Beantworte diese Fragen über Kroatien:

\begin{enumerate}
    \item Was ist die Hauptstadt von Kroatien? \underline{\hspace{6cm}}
    \item Wie viele Inseln hat Kroatien? \underline{\hspace{6cm}}
    \item Was sind die Farben der kroatischen Flagge? \underline{\hspace{6cm}}
    \item Nenne zwei berühmte kroatische Städte: \underline{\hspace{6cm}}
    \item Welches Tier ist das Nationalsymbol? \underline{\hspace{6cm}}
    \item Wer hat die Krawatte erfunden? \underline{\hspace{6cm}}
\end{enumerate}

\vspace{2cm}

\textbf{Übung 7: Kreative Aktivität}

A) Male die kroatische Flagge und beschrifte die Farben auf Kroatisch.

\vspace{5cm}

B) Erstelle ein Mini-Poster über Kroatien mit 5 lustigen Fakten, die du gelernt hast. Verwende Kroatisch und Deutsch.

\vspace{5cm}

\textbf{Übung 8: Hörübung}

\textit{Höre dir die Audiotracks 1-15 an und vervollständige Folgendes:}

\begin{enumerate}
    \item Höre Track 11 und schreibe alle Grüße auf, die du hörst:
    
    \vspace{2cm}
    
    \item Höre Track 15 und schreibe die Zahlen auf Kroatisch, wie du sie hörst:
    
    \vspace{2cm}
    
    \item Höre Dialoge 12-14 und beantworte: Wie heißen die Personen in den Dialogen?
    
    \vspace{2cm}
\end{enumerate}
\end{exercise}

\section{Zusammenfassung}

In dieser Einheit hast du gelernt:
\begin{itemize}
    \item Das kroatische Alphabet mit 30 Buchstaben (einschließlich vollständiger Alphabettabelle)
    \item Besondere kroatische Buchstaben: č, ć, dž, đ, š, ž, lj, nj (alle 8 besonderen Buchstaben)
    \item Ausführlicher Ausspracheführer für Vokale und Konsonanten
    \item Grundlegende Begrüßungen und höfliche Ausdrücke für verschiedene Tageszeiten
    \item Wie man sich vorstellt und einfache Gespräche führt
    \item Interaktive Dialoge auf Kroatisch mit Übersetzungen
    \item Zahlen 1-20 mit Zahlenmustern und Übungsaktivitäten
    \item Umfassende Fakten über Kroatien: Geographie, Flagge, Städte
    \item 10 lustige Fakten über Kroatien einschließlich Kultur, Geschichte und Sport
    \item Kroatische nationale Symbole (Tier, Blume, Baum)
\end{itemize}

\vspace{1cm}

\textbf{Gut gemacht, Einheit 1 abgeschlossen! Du bist jetzt bereit für Einheit 2!}

\textit{Audio-Erinnerung: Überprüfe die Tracks 1-15, um dein Lernen zu festigen}
