\documentclass[11pt,a4paper,twoside]{book}

% Language and encoding (for XeLaTeX)
\usepackage{fontspec}
\usepackage[german]{babel}

% Page layout
\usepackage[margin=2.5cm, headheight=14pt]{geometry}
\usepackage{fancyhdr}

% Graphics and colors
\usepackage{graphicx}
\usepackage{xcolor}
\usepackage{tikz}

% Lists and tables
\usepackage{enumitem}
\usepackage{booktabs}
\usepackage{longtable}

% Boxes and frames
\usepackage{tcolorbox}
\tcbuselibrary{most}

% Fonts and typography
\usepackage{lmodern}
\usepackage{microtype}

% Hyperlinks
\usepackage{hyperref}
\hypersetup{
    colorlinks=true,
    linkcolor=blue,
    filecolor=magenta,
    urlcolor=cyan,
    pdftitle={Hrvatski za Djecu - Kroatisch für Kinder},
    pdfauthor={Werner J.},
}

% Custom colors
\definecolor{croatianblue}{RGB}{0,91,165}
\definecolor{croatianred}{RGB}{255,0,0}
\definecolor{lightblue}{RGB}{230,240,255}
\definecolor{lightgreen}{RGB}{230,255,230}
\definecolor{lightyellow}{RGB}{255,250,205}

% Custom boxes
\newtcolorbox{vocabulary}{
    colback=lightblue,
    colframe=croatianblue,
    title=Vokabular / Wortschatz,
    fonttitle=\bfseries,
    breakable,
}

\newtcolorbox{grammar}{
    colback=lightgreen,
    colframe=green!60!black,
    title=Gramatika / Grammatik,
    fonttitle=\bfseries,
    breakable,
}

\newtcolorbox{culture}{
    colback=lightyellow,
    colframe=orange!80!black,
    title=Kultura / Kultur,
    fonttitle=\bfseries,
    breakable,
}

\newtcolorbox{exercise}{
    colback=white,
    colframe=croatianred,
    title=Vježba / Übung,
    fonttitle=\bfseries,
    breakable,
}

% Header and footer
\pagestyle{fancy}
\fancyhf{}
\fancyhead[LE]{\leftmark}
\fancyhead[RO]{\rightmark}
\fancyfoot[C]{\thepage}

% Title information
\title{\Huge\textbf{Hrvatski za Djecu}\\[0.5cm]
\Large Kroatisch für deutschsprachige Kinder\\[0.3cm]
\large Ein Einjahres-Kurs für Kinder von 8-11 Jahren}
\author{Werner J.}
\date{\today}

\begin{document}

% Title page
\maketitle

% Copyright page
\thispagestyle{empty}
\vspace*{\fill}
\begin{center}
\textcopyright{} 2025 Werner J.\\[0.5cm]
This work is licensed under the GNU General Public License v3.0\\
See LICENSE file for details.\\[1cm]
\textit{Hrvatski za Djecu - Kroatisch für Kinder}\\
Ein umfassender Einjahres-Kurs für deutschsprachige Schüler im Alter von 8-11 Jahren.
\end{center}
\vspace*{\fill}
\clearpage

% Table of contents
\tableofcontents
\clearpage

% Preface
\chapter*{Vorwort}
\addcontentsline{toc}{chapter}{Vorwort}

\section*{An die Schüler}
Willkommen bei \textit{Hrvatski za Djecu}! Dieses Buch führt dich durch eine spannende Reise beim Lernen von Kroatisch, einer wunderschönen südslawischen Sprache, die von Millionen Menschen in Kroatien und auf der ganzen Welt gesprochen wird.

\section*{An Lehrer und Eltern}
Dieser Kurs ist für Schüler der 5. Klasse des Gymnasiums (8-11 Jahre) mit Deutsch als Muttersprache konzipiert. Der Lehrplan umfasst nicht nur Sprachkenntnisse, sondern auch kroatische Kultur, Geschichte, Traditionen, Geographie und bemerkenswerte Persönlichkeiten. Jede Einheit ist für etwa drei Wochen Unterricht konzipiert.

\section*{Wie man dieses Buch benutzt}
\begin{itemize}
    \item Jedes Kapitel stellt eine Einheit dar (3 Wochen Unterricht)
    \item Vokabular wird mit deutschen Übersetzungen präsentiert
    \item Grammatikerklärungen werden auf Kroatisch und Deutsch bereitgestellt
    \item Kulturelle Abschnitte stellen das kroatische Leben und die Traditionen vor
    \item Übungen werden am Ende jedes Abschnitts bereitgestellt
    \item Audiomaterialien (im Text referenziert) sollten zusammen mit diesem Buch verwendet werden
\end{itemize}

\clearpage

% Main content - Include chapter files
\chapter{Willkommen zu Kroatisch!}

\section{Einführung}
Willkommen zu deiner ersten Einheit auf Kroatisch! In dieser Einheit lernst du die Grundlagen der kroatischen Sprache, einschließlich des Alphabets, der Aussprache und einfacher Begrüßungen.

\section{Das kroatische Alphabet}

\begin{vocabulary}
Das kroatische Alphabet hat 30 Buchstaben. Es verwendet das lateinische Alphabet (wie Deutsch).

\subsection*{Vollständiges kroatisches Alphabet}

\begin{center}
\begin{tabular}{|c|c|c|c|c|c|}
\hline
A a & B b & C c & Č č & Ć ć & D d \\
\hline
Dž dž & Đ đ & E e & F f & G g & H h \\
\hline
I i & J j & K k & L l & Lj lj & M m \\
\hline
N n & Nj nj & O o & P p & R r & S s \\
\hline
Š š & T t & U u & V v & Z z & Ž ž \\
\hline
\end{tabular}
\end{center}

\subsection*{Besondere kroatische Buchstaben}

Hier sind die besonderen Buchstaben, die es im Deutschen nicht gibt:

\begin{center}
\begin{tabular}{clll}
\toprule
Buchstabe & Name & Aussprache & Beispiel \\
\midrule
Č, č & če & wie \glqq tsch\grqq{} in \glqq Tschüss\grqq{} & \textbf{č}okolada (Schokolade) \\
Ć, ć & će & weicherer \glqq tsch\grqq{}-Laut & ma\textbf{ć}ka (Katze) \\
Dž, dž & dže & wie \glqq dsch\grqq{} in \glqq Dschungel\grqq{} & \textbf{dž}em (Marmelade) \\
Đ, đ & đe & weicherer \glqq dsch\grqq{}-Laut & \textbf{đ}ak (Schüler) \\
Š, š & še & wie \glqq sch\grqq{} in \glqq Schule\grqq{} & \textbf{š}kola (Schule) \\
Ž, ž & že & stimmhaftes \glqq sch\grqq{} & \textbf{ž}ena (Frau) \\
Lj, lj & elj & palatalisiertes \glqq l\grqq{} & \textbf{lj}ubav (Liebe) \\
Nj, nj & enj & palatalisiertes \glqq n\grqq{} & ko\textbf{nj} (Pferd) \\
\bottomrule
\end{tabular}
\end{center}

\textit{Note: Audio files for pronunciation are available at [audio reference placeholder]}

\textit{Hinweis: Audiodateien zur Aussprache sind verfügbar unter [Audio-Referenz Platzhalter]}
\end{vocabulary}

\section{Ausspracheführer}

\begin{grammar}
\subsection*{Vokale}

Kroatisch hat 5 Vokale, und sie werden immer gleich ausgesprochen:

\begin{center}
\begin{tabular}{clll}
\toprule
Buchstabe & Laut & Wie im Deutschen & Beispiel \\
\midrule
A a & \glqq ah\grqq{} & Mann & \textbf{a}nanas (Ananas) \\
E e & \glqq eh\grqq{} & Bett & \textbf{e}lefant (Elefant) \\
I i & \glqq ee\grqq{} & Lied & r\textbf{i}ba (Fisch) \\
O o & \glqq oh\grqq{} & Boot & \textbf{o}ko (Auge) \\
U u & \glqq oo\grqq{} & Fuß & \textbf{u}l (Öl) \\
\bottomrule
\end{tabular}
\end{center}

\subsection*{Konsonanten}

Die meisten Konsonanten klingen ähnlich wie im Deutschen, aber beachte diese:

\begin{itemize}
    \item \textbf{C} - immer wie \glqq ts\grqq{} in \glqq Zeit\grqq{} (niemals wie \glqq k\grqq{} oder \glqq s\grqq{})
    \item \textbf{J} - immer wie \glqq j\grqq{} in \glqq ja\grqq{}
    \item \textbf{R} - gerolltes \glqq r\grqq{} (wie im Spanischen oder Italienischen)
    \item \textbf{H} - wie deutsches \glqq ch\grqq{} in \glqq ach\grqq{} (NICHT wie das \glqq h\grqq{} in \glqq Haus\grqq{})
\end{itemize}

\subsection*{Wichtige Ausspracheregeln}

\begin{enumerate}
    \item Jeder Buchstabe wird ausgesprochen - es gibt keine stummen Buchstaben!
    \item Die Betonung liegt normalerweise auf der ersten Silbe
    \item Doppelkonsonanten werden als Einzellaute ausgesprochen
\end{enumerate}

\textit{Audio examples: Listen to tracks 1-10 for alphabet pronunciation}

\textit{Audiobeispiele: Höre dir die Tracks 1-10 für die Alphabet-Aussprache an}
\end{grammar}

\section{Grundlegende Grüße}

\begin{vocabulary}
\subsection*{Alltägliche Grüße}

\begin{center}
\begin{tabular}{lll}
\toprule
Kroatisch & Deutsch & Wann zu verwenden \\
\midrule
Bok! / Bog! & Hallo! & Informell, jederzeit \\
Dobro jutro! & Guten Morgen! & Morgen (bis 10 Uhr) \\
Dobar dan! & Guten Tag! & Tag (10-18 Uhr) \\
Dobra večer! & Guten Abend! & Abend (nach 18 Uhr) \\
Laku noć! & Gute Nacht! & Vor dem Schlafengehen \\
Doviđenja! & Auf Wiedersehen! & Abschied (formell) \\
Bok! / Ćao! & Tschüss! & Abschied (informell) \\
\bottomrule
\end{tabular}
\end{center}

\subsection*{Höfliche Ausdrücke}

\begin{center}
\begin{tabular}{ll}
\toprule
Kroatisch & Deutsch \\
\midrule
Hvala! & Danke! \\
Hvala lijepo! & Vielen Dank! \\
Molim! & Bitte! \\
Molim te & Bitte (informell) \\
Molim vas & Bitte (formell) \\
Oprosti! & Entschuldigung! (informell) \\
Oprostite! & Entschuldigung! (formell) \\
Nema problema! & Kein Problem! \\
Izvini! & Verzeihung! (informell) \\
Izvinite! & Verzeihung! (formell) \\
\bottomrule
\end{tabular}
\end{center}

\subsection*{Grundlegende Antworten}

\begin{center}
\begin{tabular}{ll}
\toprule
Kroatisch & Deutsch \\
\midrule
Da & Ja \\
Ne & Nein \\
U redu & In Ordnung / OK \\
Dobro & Gut \\
Super! & Super! \\
Odlično! & Ausgezeichnet! \\
\bottomrule
\end{tabular}
\end{center}

\textit{Audio: Listen to track 11 for greeting pronunciations}

\textit{Audio: Höre Track 11 für die Aussprache der Grüße}
\end{vocabulary}

\section{Interaktive Dialoge}

\begin{culture}
\subsection*{Dialog 1: Erstes Treffen}

\textbf{Luka:} Bok! Ja sam Luka. \\
\textit{Hallo! Ich bin Luka.}

\textbf{Ana:} Bok, Luka! Ja sam Ana. Drago mi je! \\
\textit{Hallo, Luka! Ich bin Ana. Schön, dich kennenzulernen!}

\textbf{Luka:} Drago mi je! Koliko imaš godina? \\
\textit{Schön, dich kennenzulernen! Wie alt bist du?}

\textbf{Ana:} Imam devet godina. A ti? \\
\textit{Ich bin neun Jahre alt. Und du?}

\textbf{Luka:} Imam deset godina. \\
\textit{Ich bin zehn Jahre alt.}

\subsection*{Dialog 2: Am Morgen}

\textbf{Mama:} Dobro jutro, Ivan! \\
\textit{Guten Morgen, Ivan!}

\textbf{Ivan:} Dobro jutro, mama! \\
\textit{Guten Morgen, Mama!}

\textbf{Mama:} Kako si? \\
\textit{Wie geht es dir?}

\textbf{Ivan:} Dobro sam, hvala! A ti? \\
\textit{Mir geht es gut, danke! Und dir?}

\textbf{Mama:} I ja sam dobro, hvala! \\
\textit{Mir geht es auch gut, danke!}

\subsection*{Dialog 3: In der Schule}

\textbf{Učitelj:} Dobar dan, djeco! \\
\textit{Guten Tag, Kinder!}

\textbf{Djeca:} Dobar dan, učitelju! \\
\textit{Guten Tag, Herr Lehrer!}

\textbf{Učitelj:} Kako ste danas? \\
\textit{Wie geht es euch heute?}

\textbf{Marta:} Dobro smo, hvala! \\
\textit{Uns geht es gut, danke!}

\textit{Audio: Listen to tracks 12-14 for dialogue practice}

\textit{Audio: Höre Tracks 12-14 für Dialog-Übungen}
\end{culture}

\section{Sich vorstellen}

\begin{grammar}
Um dich auf Kroatisch vorzustellen:

\begin{itemize}
    \item \textbf{Ja sam [name]} - Ich bin [Name]
    \item \textbf{Zovem se [name]} - Ich heiße [Name]
    \item \textbf{Imam [age] godina} - Ich bin [Alter] Jahre alt
\end{itemize}
\end{grammar}

\textbf{Beispiel:}
\begin{itemize}
    \item Bok! Ja sam Ana. Imam devet godina.
    \item Hallo! Ich bin Ana. Ich bin neun Jahre alt.
\end{itemize}

\section{Zahlen 1-20}

\begin{vocabulary}
Zahlen lernen macht Spaß und ist nützlich! Übe, alles um dich herum zu zählen.

\begin{center}
\begin{tabular}{llll}
\toprule
\multicolumn{2}{c}{Kroatisch} & \multicolumn{2}{c}{Deutsch} \\
\midrule
1 - jedan & 11 - jedanaest & 1 - eins & 11 - elf \\
2 - dva & 12 - dvanaest & 2 - zwei & 12 - zwölf \\
3 - tri & 13 - trinaest & 3 - drei & 13 - dreizehn \\
4 - četiri & 14 - četrnaest & 4 - vier & 14 - vierzehn \\
5 - pet & 15 - petnaest & 5 - fünf & 15 - fünfzehn \\
6 - šest & 16 - šesnaest & 6 - sechs & 16 - sechzehn \\
7 - sedam & 17 - sedamnaest & 7 - sieben & 17 - siebzehn \\
8 - osam & 18 - osamnaest & 8 - acht & 18 - achtzehn \\
9 - devet & 19 - devetnaest & 9 - neun & 19 - neunzehn \\
10 - deset & 20 - dvadeset & 10 - zehn & 20 - zwanzig \\
\bottomrule
\end{tabular}
\end{center}

\subsection*{Zahlenmuster}

Beachte das Muster für 11-19:

\begin{itemize}
    \item Die Zahlen 11-19 werden gebildet, indem \textbf{-naest} an die Basiszahl angehängt wird
    \item 11: jedan + naest = jedanaest
    \item 12: dva + naest = dvanaest
    \item 13: tri + naest = trinaest
    \item usw.
\end{itemize}

\textit{Audio: Listen to track 15 for number pronunciation}

\textit{Audio: Höre Track 15 für die Aussprache der Zahlen}
\end{vocabulary}

\section{Zahlen-Übungsaktivitäten}

\begin{exercise}
\subsection*{Aktivität 1: Zählspiel}

Zähle diese Gegenstände und schreibe die kroatische Zahl:

\begin{enumerate}
    \item 5 Sterne (stjerne) = \underline{\hspace{3cm}} (pet)
    \item 7 Blumen (cvijeće) = \underline{\hspace{3cm}}
    \item 11 Äpfel (jabuke) = \underline{\hspace{3cm}}
    \item 15 Bücher (knjige) = \underline{\hspace{3cm}}
    \item 20 Katzen (mačke) = \underline{\hspace{3cm}}
\end{enumerate}

\subsection*{Aktivität 2: Mathematik auf Kroatisch}

Löse diese einfachen Matheaufgaben und schreibe die Antwort auf Kroatisch:

\begin{enumerate}
    \item pet + tri = \underline{\hspace{3cm}}
    \item deset - dva = \underline{\hspace{3cm}}
    \item sedam + šest = \underline{\hspace{3cm}}
    \item petnaest - pet = \underline{\hspace{3cm}}
    \item devet + jedan = \underline{\hspace{3cm}}
\end{enumerate}

\textit{Tipp: Plus = plus, Minus = minus}

\subsection*{Aktivität 3: Telefonnummern}

Auf Kroatisch sagen wir Telefonnummern Ziffer für Ziffer. Übe, diese Zahlen zu sagen:

\begin{enumerate}
    \item 385-1-234-5678: tri-osam-pet, jedan, dva-tri-četiri, pet-šest-sedam-osam
    \item Deine Telefonnummer: \underline{\hspace{8cm}}
    \item Dein Alter: Imam \underline{\hspace{3cm}} godina.
\end{enumerate}
\end{exercise}

\section{Kroatien - Grundlegende Fakten}

\begin{culture}
\textbf{Willkommen in Kroatien!}

Kroatien (Hrvatska auf Kroatisch) ist ein wunderschönes Land in Südosteuropa. Lass uns darüber lernen!

\subsection*{Schnelle Fakten}

\begin{itemize}
    \item \textbf{Offizieller Name:} Republika Hrvatska (Republik Kroatien)
    \item \textbf{Hauptstadt:} Zagreb (Bevölkerung: ca. 800.000)
    \item \textbf{Bevölkerung:} Etwa 4 Millionen Menschen
    \item \textbf{Sprache:} Hrvatski (Kroatisch)
    \item \textbf{Währung:} Euro (€) seit 2023
    \item \textbf{Lage:} Südosteuropa, Adriaküste
    \item \textbf{EU-Mitglied:} Seit 1. Juli 2013
    \item \textbf{Nachbarländer:} Slowenien, Ungarn, Serbien, Bosnien und Herzegowina, Montenegro
\end{itemize}

\subsection*{Die kroatische Flagge}

Die kroatische Flagge hat drei horizontale Streifen:
\begin{itemize}
    \item \textcolor{croatianred}{\textbf{Rot}} (oben) - steht für das Blut kroatischer Helden
    \item \textbf{Weiß} (Mitte) - steht für Frieden
    \item \textcolor{croatianblue}{\textbf{Blau}} (unten) - steht für Himmel und Meer
\end{itemize}

In der Mitte befindet sich das kroatische Wappen mit einem rot-weißen Schachbrettmuster (šahovnica).

\subsection*{Geographie}

Kroatien hat drei Hauptregionen:

\begin{enumerate}
    \item \textbf{Küstenregion (Jadranska regija):} 
    \begin{itemize}
        \item Wunderschöne Strände an der Adria
        \item Über 1.000 Inseln!
        \item Berühmte Städte: Split, Dubrovnik, Zadar, Rijeka
    \end{itemize}
    
    \item \textbf{Zentralkroatien (Središnja Hrvatska):}
    \begin{itemize}
        \item Die Hauptstadt Zagreb liegt hier
        \item Hügelige Landschaft und Landwirtschaft
    \end{itemize}
    
    \item \textbf{Ostkroatien (Slavonija):}
    \begin{itemize}
        \item Flache Ebenen
        \item Wichtig für die Landwirtschaft
    \end{itemize}
\end{enumerate}

\subsection*{Große Städte}

\begin{itemize}
    \item \textbf{Zagreb} - Die Hauptstadt, größte Stadt
    \item \textbf{Split} - Zweitgrößte Stadt, antiker römischer Palast
    \item \textbf{Dubrovnik} - \glqq Perle der Adria\grqq{}, mittelalterliche Mauern
    \item \textbf{Rijeka} - Wichtige Hafenstadt
    \item \textbf{Zadar} - Historische Stadt mit römischen Ruinen
    \item \textbf{Pula} - Berühmtes römisches Amphitheater
\end{itemize}

\textit{Fotos: Siehe Illustrationen der kroatischen Landschaften und Städte}
\end{culture}

\section{Spaßige Fakten über Kroatien}

\begin{culture}
\subsection*{Wusstest du das?}

\begin{enumerate}
    \item \textbf{Die Krawatte:} Die Krawatte wurde von kroatischen Soldaten im 17. Jahrhundert erfunden! Das Wort \glqq Krawatte\grqq{} kommt von \glqq Kroate.\grqq{}
    
    \item \textbf{Dalmatiner-Hunde:} Diese gepunkteten Hunde kamen ursprünglich aus Dalmatien, einer Region in Kroatien!
    
    \item \textbf{Wunderschöne Natur:} Kroatien hat 8 Nationalparks und 11 Naturparks, einschließlich der berühmten Plitvicer Seen.
    
    \item \textbf{Inselparadies:} Kroatien hat 1.244 Inseln, aber nur 48 sind bewohnt!
    
    \item \textbf{Antike Geschichte:} Der römische Kaiser Diokletian baute seinen Altersruhesitz in Split, der jetzt UNESCO-Weltkulturerbe ist!
    
    \item \textbf{Game of Thrones:} Die beliebte TV-Serie wurde in Dubrovnik gedreht! Die Stadt wurde zu \glqq Königsmund.\grqq{}
    
    \item \textbf{Nikola Tesla:} Der berühmte Erfinder wurde in Kroatien geboren (obwohl er später in Amerika lebte).
    
    \item \textbf{Wunderschöne Küste:} Das Adriatische Meer entlang der kroatischen Küste ist kristallklar und perfekt zum Schwimmen!
    
    \item \textbf{Musik und Tanz:} Kroatien hat viele traditionelle Volkstänze und ein einzigartiges Musikinstrument namens Tamburica.
    
    \item \textbf{Sport-Champions:} Kroatien ist berühmt für Wasserball, Handball und Fußball. Die Nationalmannschaft wurde 2018 Vizeweltmeister!
\end{enumerate}

\subsection*{Kroatische Symbole}

\begin{itemize}
    \item \textbf{Nationaltier:} Baummarder (kuna) - auch der Name der alten kroatischen Währung!
    \item \textbf{Nationalblume:} Iris croatica (Kroatische Iris)
    \item \textbf{Nationalbaum:} Eiche (hrast)
\end{itemize}

\textit{Illustrationen: Bilder der kroatischen Symbole und Wahrzeichen}
\end{culture}

\section{Übungen}

\begin{exercise}
\textbf{Übung 1: Alphabet-Übung}

A) Schreibe das vollständige kroatische Alphabet (alle 30 Buchstaben):

\vspace{2cm}

B) Kreise die 8 besonderen Buchstaben ein, die es im Deutschen nicht gibt:

\vspace{1cm}

C) Verbinde die kroatischen Wörter mit ihrer Aussprache:

\begin{tabular}{ll}
1. škola & a) \glqq Schokolade\grqq{} \\
2. čokolada & b) \glqq Schule\grqq{} \\
3. mačka & c) \glqq Marmelade\grqq{} \\
4. džem & d) \glqq Katze\grqq{} \\
\end{tabular}

\vspace{2cm}

\textbf{Übung 2: Grüße}

A) Fülle die Lücken mit dem richtigen kroatischen Gruß:

\begin{enumerate}
    \item Um 8:00 Uhr morgens sagst du: \underline{\hspace{5cm}}
    \item Um 14:00 Uhr sagst du: \underline{\hspace{5cm}}
    \item Um 20:00 Uhr sagst du: \underline{\hspace{5cm}}
    \item Vor dem Schlafengehen: \underline{\hspace{5cm}}
    \item Beim Abschied: \underline{\hspace{5cm}}
\end{enumerate}

B) Übersetze diese höflichen Ausdrücke ins Kroatische:

\begin{enumerate}
    \item Danke: \underline{\hspace{5cm}}
    \item Bitte: \underline{\hspace{5cm}}
    \item Entschuldigung: \underline{\hspace{5cm}}
    \item Kein Problem: \underline{\hspace{5cm}}
\end{enumerate}

\vspace{2cm}

\textbf{Übung 3: Vorstellung}

Schreibe eine kurze Vorstellung über dich auf Kroatisch (mindestens 5 Sätze). Füge hinzu:
- Deinen Namen
- Dein Alter
- Einen Gruß
- Wie es dir geht
- Einen Abschied

\vspace{5cm}

\textbf{Übung 4: Dialog-Übung}

Vervollständige diesen Dialog mit passenden kroatischen Sätzen:

\textbf{Marko:} Bok! Ja sam Marko. \\
\textbf{Du:} \underline{\hspace{8cm}} \\
\textbf{Marko:} Drago mi je! Koliko imaš godina? \\
\textbf{Du:} \underline{\hspace{8cm}} \\
\textbf{Marko:} Kako si danas? \\
\textbf{Du:} \underline{\hspace{8cm}} \\
\textbf{Marko:} I ja sam dobro! Doviđenja! \\
\textbf{Du:} \underline{\hspace{8cm}}

\vspace{2cm}

\textbf{Übung 5: Zahlen}

A) Schreibe diese Zahlen auf Kroatisch:

\begin{enumerate}
    \item 7: \underline{\hspace{4cm}}
    \item 13: \underline{\hspace{4cm}}
    \item 18: \underline{\hspace{4cm}}
    \item 5: \underline{\hspace{4cm}}
    \item 20: \underline{\hspace{4cm}}
\end{enumerate}

B) Übe das Zählen von 1 bis 20 auf Kroatisch. Beantworte dann diese Fragen auf Kroatisch:

\begin{itemize}
    \item Wie alt bist du? \textit{Koliko imaš godina?} \\
    Antwort: \underline{\hspace{8cm}}
    
    \item Zähle deine Finger (1-10): \\
    Antwort: \underline{\hspace{8cm}}
    
    \item Wie viele Schüler sind in deiner Klasse? \\
    Antwort: U mojoj školi ima \underline{\hspace{4cm}} učenika.
\end{itemize}

\vspace{2cm}

\textbf{Übung 6: Kroatien-Fakten}

Beantworte diese Fragen über Kroatien:

\begin{enumerate}
    \item Was ist die Hauptstadt von Kroatien? \underline{\hspace{6cm}}
    \item Wie viele Inseln hat Kroatien? \underline{\hspace{6cm}}
    \item Was sind die Farben der kroatischen Flagge? \underline{\hspace{6cm}}
    \item Nenne zwei berühmte kroatische Städte: \underline{\hspace{6cm}}
    \item Welches Tier ist das Nationalsymbol? \underline{\hspace{6cm}}
    \item Wer hat die Krawatte erfunden? \underline{\hspace{6cm}}
\end{enumerate}

\vspace{2cm}

\textbf{Übung 7: Kreative Aktivität}

A) Male die kroatische Flagge und beschrifte die Farben auf Kroatisch.

\vspace{5cm}

B) Erstelle ein Mini-Poster über Kroatien mit 5 lustigen Fakten, die du gelernt hast. Verwende Kroatisch und Deutsch.

\vspace{5cm}

\textbf{Übung 8: Hörübung}

\textit{Höre dir die Audiotracks 1-15 an und vervollständige Folgendes:}

\begin{enumerate}
    \item Höre Track 11 und schreibe alle Grüße auf, die du hörst:
    
    \vspace{2cm}
    
    \item Höre Track 15 und schreibe die Zahlen auf Kroatisch, wie du sie hörst:
    
    \vspace{2cm}
    
    \item Höre Dialoge 12-14 und beantworte: Wie heißen die Personen in den Dialogen?
    
    \vspace{2cm}
\end{enumerate}
\end{exercise}

\section{Zusammenfassung}

In dieser Einheit hast du gelernt:
\begin{itemize}
    \item Das kroatische Alphabet mit 30 Buchstaben (einschließlich vollständiger Alphabettabelle)
    \item Besondere kroatische Buchstaben: č, ć, dž, đ, š, ž, lj, nj (alle 8 besonderen Buchstaben)
    \item Ausführlicher Ausspracheführer für Vokale und Konsonanten
    \item Grundlegende Begrüßungen und höfliche Ausdrücke für verschiedene Tageszeiten
    \item Wie man sich vorstellt und einfache Gespräche führt
    \item Interaktive Dialoge auf Kroatisch mit Übersetzungen
    \item Zahlen 1-20 mit Zahlenmustern und Übungsaktivitäten
    \item Umfassende Fakten über Kroatien: Geographie, Flagge, Städte
    \item 10 lustige Fakten über Kroatien einschließlich Kultur, Geschichte und Sport
    \item Kroatische nationale Symbole (Tier, Blume, Baum)
\end{itemize}

\vspace{1cm}

\textbf{Gut gemacht, Einheit 1 abgeschlossen! Du bist jetzt bereit für Einheit 2!}

\textit{Audio-Erinnerung: Überprüfe die Tracks 1-15, um dein Lernen zu festigen}

\chapter{My Family and I / Meine Familie und ich}

\section{Introduction / Einführung}
% Content to be developed: Family vocabulary, personal pronouns, verb "biti" (to be)
% Cultural focus: Croatian family traditions and naming customs

\section{Family Members / Familienmitglieder}
% Vocabulary: mama, tata, brat, sestra, djed, baka, etc.

\section{Personal Pronouns / Personalpronomen}
% Grammar: ja, ti, on, ona, mi, vi, oni/one

\section{The Verb "Biti" (To Be) / Das Verb "Biti" (Sein)}
% Grammar: ja sam, ti si, on/ona je, mi smo, vi ste, oni/one su

\section{Possessive Pronouns / Possessivpronomen}
% Grammar: moj, tvoj, njegov, njen, naš, vaš, njihov

\section{Croatian Family Traditions / Kroatische Familientraditionen}
% Cultural content to be developed

\section{Exercises / Übungen}
% Exercises to be developed

\section{Summary / Zusammenfassung}
% Summary to be developed

\chapter{Colors and the World Around Us / Farben und die Welt um uns herum}

\section{Introduction / Einführung}
% Content to be developed: Colors, adjectives, gender agreement, numbers 21-100

\section{Colors / Farben}
% Vocabulary: crvena, plava, zelena, žuta, etc.

\section{Adjectives / Adjektive}
% Grammar: velik, mali, lijep, ružan, dobar, loš

\section{Gender Agreement / Geschlechtsübereinstimmung}
% Grammar: masculine, feminine, neuter endings

\section{Numbers 21-100 / Zahlen 21-100}
% Vocabulary: counting from 21 to 100

\section{Croatian Flag and Symbols / Kroatische Flagge und Symbole}
% Cultural content: flag, coat of arms, šahovnica

\section{Natural Beauty of Croatia / Naturschönheiten Kroatiens}
% Cultural content: Plitvice Lakes, Adriatic coast

\section{Exercises / Übungen}
% Exercises to be developed

\section{Summary / Zusammenfassung}
% Summary to be developed

\chapter{School and Learning / Schule und Lernen}

\section{Introduction / Einführung}
% Content to be developed: School vocabulary, verbs, days of the week, prepositions

\section{School Vocabulary / Schulvokabular}
% Vocabulary: škola, učitelj, učenik, knjiga, olovka, bilježnica, etc.

\section{Important Verbs / Wichtige Verben}
% Grammar: imati (to have), učiti (to learn), pisati (to write), čitati (to read)

\section{Days of the Week / Wochentage}
% Vocabulary: ponedjeljak, utorak, srijeda, četvrtak, petak, subota, nedjelja

\section{Prepositions / Präpositionen}
% Grammar: u, na, iz, do, za, bez

\section{The Croatian School System / Das kroatische Schulsystem}
% Cultural content to be developed

\section{Exercises / Übungen}
% Exercises to be developed

\section{Summary / Zusammenfassung}
% Summary to be developed

\chapter{My Home and My City / Mein Zuhause und meine Stadt}

\section{Introduction / Einführung}
% Content to be developed: House/room vocabulary, locative case, question words

\section{House and Rooms / Haus und Zimmer}
% Vocabulary: kuća, stan, soba, kuhinja, kupaonica, dnevna soba, etc.

\section{Furniture / Möbel}
% Vocabulary: stol, stolica, krevet, ormar, polica, etc.

\section{The Locative Case / Der Lokativ}
% Grammar: basic introduction to locative case (gdje? - where?)

\section{Question Words / Fragewörter}
% Grammar: što, tko, gdje, kada, zašto, kako, koji

\section{Croatian Cities / Kroatische Städte}
% Cultural content: Zagreb, Split, Dubrovnik, Rijeka, Zadar

\section{Traditional Architecture / Traditionelle Architektur}
% Cultural content to be developed

\section{Exercises / Übungen}
% Exercises to be developed

\section{Summary / Zusammenfassung}
% Summary to be developed

\chapter{Food and Meals / Essen und Mahlzeiten}

\section{Introduction / Einführung}
% Content to be developed: Food vocabulary, meal times, verbs, accusative case

\section{Food Vocabulary / Lebensmittelvokabular}
% Vocabulary: kruh, meso, riba, povrće, voće, sir, mlijeko, etc.

\section{Meals / Mahlzeiten}
% Vocabulary: doručak, ručak, večera, užina

\section{Verbs for Eating and Drinking / Verben für Essen und Trinken}
% Grammar: jesti, piti, kuhati, voleti

\section{The Accusative Case / Der Akkusativ}
% Grammar: basic introduction to accusative case (što? - what?)

\section{Croatian Cuisine / Kroatische Küche}
% Cultural content: traditional dishes, regional specialties

\section{Table Manners and Customs / Tischmanieren und Bräuche}
% Cultural content to be developed

\section{Exercises / Übungen}
% Exercises to be developed

\section{Summary / Zusammenfassung}
% Summary to be developed

\chapter{Seasons and Weather / Jahreszeiten und Wetter}

\section{Introduction / Einführung}
% Content to be developed: Seasons, weather vocabulary, months

\section{Seasons / Jahreszeiten}
% Vocabulary: proljeće, ljeto, jesen, zima

\section{Weather Vocabulary / Wettervokabular}
% Vocabulary: sunce, kiša, snijeg, vjetar, oblak, magla, etc.

\section{Months of the Year / Monate des Jahres}
% Vocabulary: siječanj, veljača, ožujak, travanj, svibanj, lipanj, etc.

\section{Temperature and Weather Expressions / Temperatur und Wetterausdrücke}
% Phrases: Toplo je, Hladno je, Pada kiša, Sviće sunce, etc.

\section{Croatian Climate / Kroatisches Klima}
% Cultural content: climate zones, geographical features

\section{Seasonal Traditions / Saisonale Traditionen}
% Cultural content to be developed

\section{Exercises / Übungen}
% Exercises to be developed

\section{Summary / Zusammenfassung}
% Summary to be developed

\chapter{Hobbies and Free Time / Hobbys und Freizeit}

\section{Introduction / Einführung}
% Content to be developed: Hobby vocabulary, verbs, time expressions, frequency adverbs

\section{Hobby Vocabulary / Hobbyvokabular}
% Vocabulary: sport, nogomet, košarka, glazba, pjevanje, ples, čitanje, etc.

\section{Verbs for Hobbies / Verben für Hobbys}
% Grammar: igrati, čitati, gledati, slušati, pjevati, plesati

\section{Time Expressions / Zeitausdrücke}
% Grammar: sada, prije, poslije, jučer, danas, sutra

\section{Frequency Adverbs / Häufigkeitsadverbien}
% Grammar: uvijek, često, ponekad, rijetko, nikad

\section{Croatian Sports / Kroatischer Sport}
% Cultural content: football, water polo, handball, tennis

\section{Famous Croatian Athletes / Berühmte kroatische Athleten}
% Cultural content to be developed

\section{Exercises / Übungen}
% Exercises to be developed

\section{Summary / Zusammenfassung}
% Summary to be developed

\chapter{Croatian History and Traditions / Kroatische Geschichte und Traditionen}

\section{Introduction / Einführung}
% Content to be developed: Past tense, historical vocabulary

\section{Past Tense Introduction / Einführung in die Vergangenheit}
% Grammar: basic past tense formation

\section{Historical Vocabulary / Historisches Vokabular}
% Vocabulary: kralj, kraljica, rat, mir, povijest, etc.

\section{Croatian History Overview / Überblick über die kroatische Geschichte}
% Cultural content: Roman times, medieval period, modern history

\section{Croatian Kings and Historical Figures / Kroatische Könige und historische Figuren}
% Cultural content: King Tomislav, King Zvonimir, etc.

\section{National Symbols / Nationale Symbole}
% Cultural content: coat of arms, flag, anthem

\section{Exercises / Übungen}
% Exercises to be developed

\section{Summary / Zusammenfassung}
% Summary to be developed

\chapter{Holidays and Celebrations / Feiertage und Feste}

\section{Introduction / Einführung}
% Content to be developed: Holiday vocabulary, future tense, invitation phrases

\section{Holiday Vocabulary / Feiertagsvokabular}
% Vocabulary: Božić, Uskrs, Nova godina, rođendan, etc.

\section{Future Tense Introduction / Einführung in die Zukunft}
% Grammar: basic future tense with "ću" auxiliary

\section{Invitation and Celebration Phrases / Einladungs- und Festphrasen}
% Phrases: Pozivam te, Dođi na proslavu, Sretan rođendan, etc.

\section{Croatian Holidays / Kroatische Feiertage}
% Cultural content: Christmas, Easter, New Year, Independence Day

\section{St. Blaise Day and Patron Saints / Sv. Vlaho und Schutzheilige}
% Cultural content to be developed

\section{Folklore and Traditional Costumes / Folklore und traditionelle Kostüme}
% Cultural content to be developed

\section{Exercises / Übungen}
% Exercises to be developed

\section{Summary / Zusammenfassung}
% Summary to be developed

\chapter{Animals and Nature / Tiere und Natur}

\section{Introduction / Einführung}
% Content to be developed: Animal vocabulary, nature vocabulary, verbs of movement, genitive case

\section{Animals / Tiere}
% Vocabulary: pas, mačka, ptica, riba, konj, krava, ovca, etc.

\section{Nature Vocabulary / Naturvokabular}
% Vocabulary: drvo, cvijet, planina, more, rijeka, jezero, šuma, etc.

\section{Verbs of Movement / Bewegungsverben}
% Grammar: ići, hodati, trčati, plivati, letjeti

\section{The Genitive Case / Der Genitiv}
% Grammar: basic introduction to genitive case (čiji? - whose? od čega? - from what?)

\section{Croatian National Parks / Kroatische Nationalparks}
% Cultural content: Plitvice, Krka, Paklenica, Kornati, etc.

\section{Croatian Flora and Fauna / Kroatische Flora und Fauna}
% Cultural content to be developed

\section{Exercises / Übungen}
% Exercises to be developed

\section{Summary / Zusammenfassung}
% Summary to be developed

\chapter{Putting It All Together / Alles zusammenfügen}

\section{Introduction / Einführung}
% Content to be developed: Review of all content, complex conversations, reading, writing

\section{Grammar Review / Grammatikwiederholung}
% Review of all grammar topics from Units 1-11

\section{Vocabulary Review / Vokabelwiederholung}
% Review of all vocabulary topics from Units 1-11

\section{Reading Comprehension / Leseverständnis}
% Simple texts and stories in Croatian

\section{Writing Practice / Schreibübung}
% Guided writing exercises and compositions

\section{Conversation Practice / Gesprächsübung}
% Dialogues and conversation scenarios

\section{Famous Croatian People / Berühmte Kroaten}
% Cultural content: Nikola Tesla, Marco Polo, Ruđer Bošković, etc.

\section{Modern Croatia / Modernes Kroatien}
% Cultural content: EU membership, tourism, current events

\section{Croatian Contributions to World Culture / Kroatische Beiträge zur Weltkultur}
% Cultural content to be developed

\section{Final Project Ideas / Ideen für Abschlussprojekte}
% Suggestions for final presentations or projects

\section{Summary / Zusammenfassung}
% Course conclusion and next steps


% Appendices
\appendix
\chapter{Ausspracheführer}
% Pronunciation Guide for Croatian Language
% Comprehensive guide comparing Croatian sounds with German pronunciation

\section{Einführung}

Dieser Ausspracheführer bietet detaillierte Informationen über kroatische Laute, mit besonderem Fokus auf Buchstaben und Laute, die sich vom Deutschen unterscheiden. Jeder Laut enthält IPA-Transkription (Internationales Phonetisches Alphabet), Vergleich mit Deutsch, häufige Fehler und Übungswörter.

\subsection{Audio-Referenzen}

Audiodateien für alle Aussprachebeispiele finden Sie in den begleitenden digitalen Materialien:
\begin{itemize}
    \item \texttt{audio/pronunciation/letters/} - Einzelne Buchstabenlaute
    \item \texttt{audio/pronunciation/words/} - Übungswörter
    \item \texttt{audio/pronunciation/sentences/} - Beispielsätze
\end{itemize}

\section{Das kroatische Alphabet}

Das kroatische Alphabet (kroatisch: \textit{hrvatska abeceda} oder \textit{gajica}) hat 30 Buchstaben. Es verwendet die lateinische Schrift mit mehreren speziellen diakritischen Zeichen.

\subsection{Vollständiges Alphabet}

\begin{center}
\begin{tabular}{lll}
\toprule
\textbf{Buchstabe} & \textbf{IPA} & \textbf{Audio-Referenz} \\
\midrule
A, a & [a] & \texttt{audio/pronunciation/letters/a.mp3} \\
B, b & [b] & \texttt{audio/pronunciation/letters/b.mp3} \\
C, c & [ts] & \texttt{audio/pronunciation/letters/c.mp3} \\
Č, č & [tʃ] & \texttt{audio/pronunciation/letters/č.mp3} \\
Ć, ć & [tɕ] & \texttt{audio/pronunciation/letters/ć.mp3} \\
D, d & [d] & \texttt{audio/pronunciation/letters/d.mp3} \\
Dž, dž & [dʒ] & \texttt{audio/pronunciation/letters/dž.mp3} \\
Đ, đ & [dʑ] & \texttt{audio/pronunciation/letters/đ.mp3} \\
E, e & [e] & \texttt{audio/pronunciation/letters/e.mp3} \\
F, f & [f] & \texttt{audio/pronunciation/letters/f.mp3} \\
G, g & [ɡ] & \texttt{audio/pronunciation/letters/g.mp3} \\
H, h & [x] & \texttt{audio/pronunciation/letters/h.mp3} \\
I, i & [i] & \texttt{audio/pronunciation/letters/i.mp3} \\
J, j & [j] & \texttt{audio/pronunciation/letters/j.mp3} \\
K, k & [k] & \texttt{audio/pronunciation/letters/k.mp3} \\
L, l & [l] & \texttt{audio/pronunciation/letters/l.mp3} \\
Lj, lj & [ʎ] & \texttt{audio/pronunciation/letters/lj.mp3} \\
M, m & [m] & \texttt{audio/pronunciation/letters/m.mp3} \\
N, n & [n] & \texttt{audio/pronunciation/letters/n.mp3} \\
Nj, nj & [ɲ] & \texttt{audio/pronunciation/letters/nj.mp3} \\
O, o & [o] & \texttt{audio/pronunciation/letters/o.mp3} \\
P, p & [p] & \texttt{audio/pronunciation/letters/p.mp3} \\
R, r & [r] & \texttt{audio/pronunciation/letters/r.mp3} \\
S, s & [s] & \texttt{audio/pronunciation/letters/s.mp3} \\
Š, š & [ʃ] & \texttt{audio/pronunciation/letters/š.mp3} \\
T, t & [t] & \texttt{audio/pronunciation/letters/t.mp3} \\
U, u & [u] & \texttt{audio/pronunciation/letters/u.mp3} \\
V, v & [ʋ] & \texttt{audio/pronunciation/letters/v.mp3} \\
Z, z & [z] & \texttt{audio/pronunciation/letters/z.mp3} \\
Ž, ž & [ʒ] & \texttt{audio/pronunciation/letters/ž.mp3} \\
\bottomrule
\end{tabular}
\end{center}

\section{Besondere kroatische Buchstaben}

Diese Buchstaben gibt es im Deutschen nicht und erfordern besondere Aufmerksamkeit.

\subsection{Č, č [tʃ] - "Tsch-Laut"}

\begin{tcolorbox}[colback=lightblue!30, colframe=croatianblue, title=\textbf{Č, č}]

\textbf{IPA:} [tʃ]

\textbf{Deutscher Vergleich:}
Ähnlich wie deutsches "tsch" in "Tschüss" oder "Deutsch".

\textbf{Wie man ausspricht:}
Platziere deine Zunge hinter deinen oberen Zähnen und lasse Luft mit einem "tsch"-Laut entweichen.

\textbf{Häufige Fehler:}
\begin{itemize}
    \item Als einfachen "c" [ts]-Laut aussprechen
    \item Nicht stark genug machen (es sollte klar und deutlich sein)
    \item Mit "ć" verwechseln (das weicher ist)
\end{itemize}

\textbf{Übungswörter:}
\begin{tabular}{lll}
\textbf{Kroatisch} & \textbf{Deutsch} & \textbf{Audio} \\
\midrule
\textbf{č}okolada & Schokolade & \texttt{words/cokolada.mp3} \\
\textbf{č}aj & Tee & \texttt{words/caj.mp3} \\
u\textbf{č}itelj & Lehrer & \texttt{words/ucitelj.mp3} \\
\textbf{č}ovjek & Mann, Mensch & \texttt{words/covjek.mp3} \\
de\textbf{č}ko & Junge & \texttt{words/decko.mp3} \\
\end{tabular}

\textbf{Übungssatz:}
\textit{Čovjek pije čaj i jede čokoladu.}
(Der Mann trinkt Tee und isst Schokolade.)
Audio: \texttt{sentences/covjek\_pije\_caj.mp3}

\end{tcolorbox}

\subsection{Ć, ć [tɕ] - "Weiches Tsch"}

\begin{tcolorbox}[colback=lightblue!30, colframe=croatianblue, title=\textbf{Ć, ć}]

\textbf{IPA:} [tɕ]

\textbf{Deutscher Vergleich:}
Ähnlich wie "č", aber weicher und palataler. Wie "tsch" sagen, während man lächelt oder mit der Zunge weiter vorne. Es gibt keine exakte deutsche Entsprechung.

\textbf{Wie man ausspricht:}
Beginne wie "č", aber platziere deine Zunge höher und weiter vorne zum harten Gaumen. Es ist ein "weicherer" Laut.

\textbf{Häufige Fehler:}
\begin{itemize}
    \item Genau wie "č" aussprechen (der häufigste Fehler!)
    \item Zu hart/rau machen
    \item Nicht zwischen "ć" und "č" unterscheiden (sie sind verschiedene Phoneme!)
\end{itemize}

\textbf{Übungswörter:}
\begin{tabular}{lll}
\textbf{Kroatisch} & \textbf{Deutsch} & \textbf{Audio} \\
\midrule
ma\textbf{ć}ka & Katze & \texttt{words/macka.mp3} \\
\textbf{ć}up & Krug & \texttt{words/cup.mp3} \\
no\textbf{ć} & Nacht & \texttt{words/noc.mp3} \\
dje\textbf{ć}a & Kinder & \texttt{words/djeca.mp3} \\
\textbf{ć}ao & Tschüss (informell) & \texttt{words/cao.mp3} \\
\end{tabular}

\textbf{Übungssatz:}
\textit{Mačka spava cijelu noć.}
(Die Katze schläft die ganze Nacht.)
Audio: \texttt{sentences/macka\_spava.mp3}

\textbf{Wichtiger Hinweis:}
Der Unterschied zwischen "č" und "ć" ist im Kroatischen entscheidend! Vergleiche:
\begin{itemize}
    \item \textbf{kuća} (Haus) vs. \textbf{kuča} (Hündin) - verschiedene Bedeutungen!
\end{itemize}

\end{tcolorbox}

\subsection{Š, š [ʃ] - "Sch-Laut"}

\begin{tcolorbox}[colback=lightblue!30, colframe=croatianblue, title=\textbf{Š, š}]

\textbf{IPA:} [ʃ]

\textbf{Deutscher Vergleich:}
Genau wie deutsches "sch" in "Schule" oder "schön".

\textbf{Wie man ausspricht:}
Das ist einfach für deutsche Sprecher! Sprich es genau wie deutsches "sch" aus.

\textbf{Häufige Fehler:}
\begin{itemize}
    \item Sehr wenige Fehler für deutsche Sprecher!
    \item Manchmal mit "s" in der Schrift verwechselt
\end{itemize}

\textbf{Übungswörter:}
\begin{tabular}{lll}
\textbf{Croatian} & \textbf{German} & \textbf{Audio} \\
\midrule
\textbf{š}kola & Schule & \texttt{words/skola.mp3} \\
\textbf{š}uma & Wald & \texttt{words/suma.mp3} \\
mi\textbf{š} & Maus & \texttt{words/mis.mp3} \\
kru\textbf{š}ka & Birne & \texttt{words/kruska.mp3} \\
\textbf{š}ecer & Zucker & \texttt{words/secer.mp3} \\
\end{tabular}

\textbf{Übungssatz:}
\textit{Škola ima veliku šumu.}
(Die Schule hat einen großen Wald.)
Audio: \texttt{sentences/skola\_ima\_sumu.mp3}

\end{tcolorbox}

\subsection{Ž, ž [ʒ] - "Stimmhaftes Sch"}

\begin{tcolorbox}[colback=lightblue!30, colframe=croatianblue, title=\textbf{Ž, ž}]

\textbf{IPA:} [ʒ]

\textbf{Deutscher Vergleich:}
Wie französisches "j" in "journal". Ähnlich wie deutsches "sch", aber stimmhaft (die Stimmbänder vibrieren).

\textbf{Wie man ausspricht:}
Beginne mit "š" [ʃ] und füge Stimme hinzu (bringe deine Stimmbänder zum Vibrieren). Lege deine Hand auf deinen Hals - du solltest Vibrationen spüren.

\textbf{Häufige Fehler:}
\begin{itemize}
    \item Als stimmloses "š" statt stimmhaftes "ž" aussprechen
    \item Mit "z" [z] verwechseln
    \item Nicht deutlich genug von "š" unterscheiden
\end{itemize}

\textbf{Übungswörter:}
\begin{tabular}{lll}
\textbf{Croatian} & \textbf{German} & \textbf{Audio} \\
\midrule
\textbf{ž}ena & Frau & \texttt{words/zena.mp3} \\
\textbf{ž}aba & Frosch & \texttt{words/zaba.mp3} \\
no\textbf{ž} & Messer & \texttt{words/noz.mp3} \\
ko\textbf{ž}a & Haut, Leder & \texttt{words/koza.mp3} \\
mu\textbf{ž} & Ehemann & \texttt{words/muz.mp3} \\
\end{tabular}

\textbf{Übungssatz:}
\textit{Žena gleda žabu u šumi.}
(Die Frau beobachtet einen Frosch im Wald.)
Audio: \texttt{sentences/zena\_gleda\_zabu.mp3}

\end{tcolorbox}

\subsection{Dž, dž [dʒ] - "Dsch-Laut"}

\begin{tcolorbox}[colback=lightblue!30, colframe=croatianblue, title=\textbf{Dž, dž}]

\textbf{IPA:} [dʒ]

\textbf{Deutscher Vergleich:}
Wie "dsch" im deutschen Lehnwort "Dschungel". Ähnlich wie die stimmhafte Version von "č".

\textbf{Wie man ausspricht:}
Beginne mit "č" [tʃ] und füge Stimme hinzu (vibriere die Stimmbänder). Es ist das stimmhafte Gegenstück zu "č".

\textbf{Häufige Fehler:}
\begin{itemize}
    \item Als "č" ohne Stimmhaftigkeit aussprechen
    \item In zwei separate Laute "d" + "ž" aufteilen
    \item Dies ist ein weniger häufiger Laut im Kroatischen, daher üben Schüler ihn möglicherweise nicht genug
\end{itemize}

\textbf{Übungswörter:}
\begin{tabular}{lll}
\textbf{Croatian} & \textbf{German} & \textbf{Audio} \\
\midrule
\textbf{dž}em & Marmelade & \texttt{words/dzem.mp3} \\
\textbf{dž}ep & Tasche & \texttt{words/dzep.mp3} \\
\textbf{dž}ungle & Dschungel & \texttt{words/dzungle.mp3} \\
\textbf{dž}ez & Jazz & \texttt{words/dzez.mp3} \\
\textbf{dž}in & Gin & \texttt{words/dzin.mp3} \\
\end{tabular}

\textbf{Übungssatz:}
\textit{Džem je u džepu.}
(Die Marmelade ist in der Tasche.)
Audio: \texttt{sentences/dzem\_u\_dzepu.mp3}

\textbf{Hinweis:}
Viele Wörter mit "dž" sind aus anderen Sprachen entlehnt.

\end{tcolorbox}

\subsection{Đ, đ [dʑ] - "Weiches Dsch"}

\begin{tcolorbox}[colback=lightblue!30, colframe=croatianblue, title=\textbf{Đ, đ}]

\textbf{IPA:} [dʑ]

\textbf{Deutscher Vergleich:}
Die stimmhafte (mit Stimmbandvibration) und weichere Version von "ć". Ähnlich wie der "d"-Laut im italienischen "Giovanni". Keine exakte deutsche Entsprechung.

\textbf{Wie man ausspricht:}
Dies ist der schwerste Laut für deutsche Sprecher! Er ist wie "dž", aber weicher und palataler (Zunge nach vorne). Denke daran als die stimmhafte Version von "ć".

\textbf{Häufige Fehler:}
\begin{itemize}
    \item Genau wie "dž" aussprechen (häufigster Fehler!)
    \item Nicht weich/palatal genug machen
    \item Mit "j" oder einfachem "d" verwechseln
    \item Nicht zwischen "đ" und "dž" unterscheiden
\end{itemize}

\textbf{Übungswörter:}
\begin{tabular}{lll}
\textbf{Croatian} & \textbf{German} & \textbf{Audio} \\
\midrule
\textbf{đ}ak & Schüler & \texttt{words/djak.mp3} \\
me\textbf{đ}ed & Bär & \texttt{words/medjed.mp3} \\
ro\textbf{đ}endan & Geburtstag & \texttt{words/rodjendan.mp3} \\
\textbf{đ}ir & Ingwer & \texttt{words/djir.mp3} \\
\textbf{đ}ubre & Dünger & \texttt{words/djubre.mp3} \\
\end{tabular}

\textbf{Übungssatz:}
\textit{Đak ima rođendan.}
(Der Schüler hat Geburtstag.)
Audio: \texttt{sentences/djak\_rodjendan.mp3}

\textbf{Wichtiger Hinweis:}
Wie bei "ć" vs. "č" ist der Unterschied zwischen "đ" und "dž" wichtig:
\begin{itemize}
    \item \textbf{đak} (Schüler) vs. \textbf{džak} (Sack) - verschiedene Bedeutungen!
\end{itemize}

\end{tcolorbox}

\subsection{C, c [ts] - "Z-Laut"}

\begin{tcolorbox}[colback=lightyellow!30, colframe=orange, title=\textbf{C, c}]

\textbf{IPA:} [ts]

\textbf{Deutscher Vergleich:}
Genau wie deutsches "z" in "Zeit" oder "Katze".

\textbf{Wie man ausspricht:}
Das ist einfach für deutsche Sprecher! Sprich es genau wie deutsches "z" aus.

\textbf{Häufige Fehler:}
\begin{itemize}
    \item Als "k" oder "s" aussprechen
    \item Schreibverwirrung: Kroatisches "c" = deutscher "z"-Laut
\end{itemize}

\textbf{Übungswörter:}
\begin{tabular}{lll}
\textbf{Croatian} & \textbf{German} & \textbf{Audio} \\
\midrule
\textbf{c}ura & Mädchen & \texttt{words/cura.mp3} \\
\textbf{c}vijet & Blume & \texttt{words/cvijet.mp3} \\
o\textbf{c}a & Vater & \texttt{words/oca.mp3} \\
u\textbf{c}a & Onkel & \texttt{words/uca.mp3} \\
me\textbf{c} & Bär & \texttt{words/mec.mp3} \\
\end{tabular}

\textbf{Übungssatz:}
\textit{Cura nosi cvijet.}
(Das Mädchen trägt eine Blume.)
Audio: \texttt{sentences/cura\_cvijet.mp3}

\end{tcolorbox}

\subsection{Lj, lj [ʎ] - "Palatales L"}

\begin{tcolorbox}[colback=lightgreen!30, colframe=green!60!black, title=\textbf{Lj, lj}]

\textbf{IPA:} [ʎ]

\textbf{Deutscher Vergleich:}
Ähnlich wie italienisches "gli" in "famiglia" oder spanisches "ll" in "llamar". Wie "l" und "j" sehr schnell zusammen sagen. Keine exakte deutsche Entsprechung.

\textbf{Wie man ausspricht:}
Platziere deine Zunge, als würdest du "l" sagen, aber drücke die Mitte deiner Zunge gegen deinen harten Gaumen. Es ist ein palatalisiertes "l".

\textbf{Häufige Fehler:}
\begin{itemize}
    \item Als zwei separate Laute "l" + "j" aussprechen
    \item Wie ein einfaches "l" klingen lassen
    \item Nicht genug palatalisieren
\end{itemize}

\textbf{Übungswörter:}
\begin{tabular}{lll}
\textbf{Croatian} & \textbf{German} & \textbf{Audio} \\
\midrule
\textbf{lj}eto & Sommer & \texttt{words/ljeto.mp3} \\
zem\textbf{lj}a & Erde, Land & \texttt{words/zemlja.mp3} \\
\textbf{lj}ubav & Liebe & \texttt{words/ljubav.mp3} \\
\textbf{lj}uljati & schaukeln & \texttt{words/ljuljati.mp3} \\
\textbf{lj}udi & Menschen & \texttt{words/ljudi.mp3} \\
\end{tabular}

\textbf{Übungssatz:}
\textit{Ljudi vole ljeto i ljubav.}
(Menschen lieben Sommer und Liebe.)
Audio: \texttt{sentences/ljudi\_ljeto.mp3}

\end{tcolorbox}

\subsection{Nj, nj [ɲ] - "Palatales N"}

\begin{tcolorbox}[colback=lightgreen!30, colframe=green!60!black, title=\textbf{Nj, nj}]

\textbf{IPA:} [ɲ]

\textbf{Deutscher Vergleich:}
Ähnlich wie spanisches "ñ" in "señor" oder italienisches "gn" in "gnocchi". Wie französisches "gn" in "cognac". Ähnlich wie "canyon" schnell sagen.

\textbf{Wie man ausspricht:}
Beginne mit "n", aber drücke die Mitte deiner Zunge gegen deinen harten Gaumen. Es ist ein palatalisiertes "n".

\textbf{Häufige Fehler:}
\begin{itemize}
    \item Als zwei separate Laute "n" + "j" aussprechen
    \item Wie ein einfaches "n" klingen lassen
    \item Nicht genug palatalisieren
\end{itemize}

\textbf{Übungswörter:}
\begin{tabular}{lll}
\textbf{Croatian} & \textbf{German} & \textbf{Audio} \\
\midrule
ko\textbf{nj} & Pferd & \texttt{words/konj.mp3} \\
k\textbf{nj}iga & Buch & \texttt{words/knjiga.mp3} \\
pje\textbf{sm}a & Lied & \texttt{words/pjesma.mp3} \\
sje\textbf{nj}a & Schatten & \texttt{words/sjena.mp3} \\
\textbf{nj}emačk\textbf{i} & deutsch & \texttt{words/njemacki.mp3} \\
\end{tabular}

\textbf{Übungssatz:}
\textit{Konj čita knjigu.}
(Das Pferd liest ein Buch.) - Albern, aber einprägsam!
Audio: \texttt{sentences/konj\_knjiga.mp3}

\end{tcolorbox}

\section{Andere wichtige Laute}

\subsection{R, r [r] - "Gerolltes R"}

\begin{tcolorbox}[colback=lightyellow!30, colframe=orange, title=\textbf{R, r}]

\textbf{IPA:} [r]

\textbf{Deutscher Vergleich:}
Das kroatische "r" ist immer gerollt, wie das "r" in einigen deutschen Dialekten (Bayrisch) oder wie italienisches/spanisches "r". Anders als das Standard-deutsche "r", das guttural ist.

\textbf{Wie man ausspricht:}
Tippe oder rolle deine Zunge gegen den Zahndamm (direkt hinter den oberen Zähnen). Lass deine Zunge vibrieren.

\textbf{Häufige Fehler:}
\begin{itemize}
    \item Deutsches gutturales "r" statt gerolltes "r" verwenden
    \item Nicht genug rollen
    \item Schwierigkeiten mit silbischem "r" (siehe besonderer Hinweis unten)
\end{itemize}

\textbf{Übungswörter:}
\begin{tabular}{lll}
\textbf{Croatian} & \textbf{German} & \textbf{Audio} \\
\midrule
\textbf{r}iba & Fisch & \texttt{words/riba.mp3} \\
\textbf{r}uka & Hand & \texttt{words/ruka.mp3} \\
k\textbf{r}uh & Brot & \texttt{words/kruh.mp3} \\
p\textbf{r}ijatelj & Freund & \texttt{words/prijatelj.mp3} \\
\textbf{r}adost & Freude & \texttt{words/radost.mp3} \\
\end{tabular}

\textbf{Besonderer Hinweis - Silbisches R:}
Im Kroatischen kann "r" eine Silbe für sich bilden, ohne Vokal! Das ist sehr ungewöhnlich.

Beispiele:
\begin{itemize}
    \item \textbf{prst} (Finger) - ausgesprochen "prrst" mit gerolltem r als Vokal
    \item \textbf{krv} (Blut) - ausgesprochen "krrv"
    \item \textbf{vrh} (Spitze, Gipfel) - ausgesprochen "vrrh"
    \item \textbf{smrt} (Tod) - ausgesprochen "smrrt"
\end{itemize}

Audio: \texttt{words/syllabic\_r\_examples.mp3}

\end{tcolorbox}

\subsection{H, h [x] - "Ch-Laut"}

\begin{tcolorbox}[colback=lightyellow!30, colframe=orange, title=\textbf{H, h}]

\textbf{IPA:} [x]

\textbf{Deutscher Vergleich:}
Wie deutsches "ch" in "Bach" oder "noch" (nach hinteren Vokalen). NICHT wie das "h" in "Haus"!

\textbf{Wie man ausspricht:}
Produziere den Laut aus dem hinteren Teil deines Rachens, wie deutsches "ch" in "ach". Das ist einfach für deutsche Sprecher!

\textbf{Häufige Fehler:}
\begin{itemize}
    \item Als schwaches "h" aussprechen
    \item Nicht guttural genug machen
    \item In manchen Regionen wird "h" kaum ausgesprochen oder weggelassen - versuche, es klar auszusprechen
\end{itemize}

\textbf{Übungswörter:}
\begin{tabular}{lll}
\textbf{Croatian} & \textbf{German} & \textbf{Audio} \\
\midrule
\textbf{h}rana & Essen & \texttt{words/hrana.mp3} \\
\textbf{h}vaditi & fangen & \texttt{words/hvaditi.mp3} \\
\textbf{h}rvatski & kroatisch & \texttt{words/hrvatski.mp3} \\
du\textbf{h} & Geist & \texttt{words/duh.mp3} \\
sni\textbf{jeh} & Schnee & \texttt{words/snijeh.mp3} \\
\end{tabular}

\textbf{Übungssatz:}
\textit{Hrvatski jezik je lijep.}
(Die kroatische Sprache ist schön.)
Audio: \texttt{sentences/hrvatski\_jezik.mp3}

\end{tcolorbox}

\section{Vokale}

Das Kroatische hat 5 Vokale: A, E, I, O, U. Sie werden konstanter ausgesprochen als im Deutschen und ähneln italienischen oder spanischen Vokalen.

\begin{tcolorbox}[colback=lightgreen!30, colframe=green!60!black, title=\textbf{Kroatische Vokale}]

\begin{center}
\begin{tabular}{lll}
\toprule
\textbf{Buchstabe} & \textbf{IPA} & \textbf{Deutscher Vergleich} \\
\midrule
A, a & [a] & Wie "a" in "Mann" \\
E, e & [e] & Wie "ä" in "Käse" oder "e" in "Meer" \\
I, i & [i] & Wie "i" in "Lied" \\
O, o & [o] & Wie "o" in "groß" \\
U, u & [u] & Wie "u" in "gut" \\
\bottomrule
\end{tabular}
\end{center}

\textbf{Wichtige Hinweise:}
\begin{itemize}
    \item Kroatische Vokale werden immer gleich ausgesprochen (keine Variationen)
    \item Sie sind reine Vokale (Monophthonge), keine Diphthonge
    \item Sie werden in unbetonten Silben nicht reduziert (anders als im Deutschen)
\end{itemize}

\textbf{Übungswörter für Vokale:}
\begin{itemize}
    \item \textbf{mama} [mama] - Mama
    \item \textbf{bebe} [bebe] - Baby (informal)
    \item \textbf{kino} [kino] - Kino
    \item \textbf{lopta} [lopta] - Ball
    \item \textbf{ruka} [ruka] - Hand
\end{itemize}

Audio: \texttt{words/vowel\_examples.mp3}

\end{tcolorbox}

\section{Betonung und Akzent}

\begin{tcolorbox}[colback=white, colframe=croatianred, title=\textbf{Wortbetonung}]

Das Kroatische hat ein komplexes System von Tönen und Akzenten, aber für Anfänger sind hier die Grundregeln:

\textbf{Grundregeln:}
\begin{itemize}
    \item Die Betonung liegt nie auf der letzten Silbe (mit seltenen Ausnahmen)
    \item In den meisten Wörtern liegt die Betonung auf der ersten oder zweiten Silbe
    \item Es gibt kurze und lange Vokale (aber weniger ausgeprägt als im Deutschen)
\end{itemize}

\textbf{Beispiele:}
\begin{itemize}
    \item \textbf{KU}ća (Haus) - Betonung auf der ersten Silbe
    \item \textbf{ŠKO}la (Schule) - Betonung auf der ersten Silbe
    \item pri\textbf{JA}telj (Freund) - Betonung auf der zweiten Silbe
\end{itemize}

\textbf{Für Anfänger:}
Mache dir anfangs nicht zu viele Gedanken über das komplexe Tonsystem. Konzentriere dich auf:
\begin{enumerate}
    \item Niemals die letzte Silbe betonen
    \item Muttersprachlern zuhören
    \item Mit Audiomaterialien üben
\end{enumerate}

\end{tcolorbox}

\section{Häufige Ausspracheprobleme}

\subsection{Zusammenfassung häufiger Fehler}

\begin{enumerate}
    \item \textbf{Č vs. Ć}: Die schwierigste Unterscheidung! Ć ist weicher und weiter vorne.
    \item \textbf{Dž vs. Đ}: Ähnlich ist đ weicher als dž.
    \item \textbf{Gerolltes R}: Deutsche Sprecher müssen lernen, das "r" zu rollen.
    \item \textbf{H-Laut}: Muss wie deutsches "ch" in "Bach" ausgesprochen werden, nicht weggelassen.
    \item \textbf{C-Laut}: Denke daran, dass es immer [ts] ist wie deutsches "z", niemals [k] oder [s].
    \item \textbf{Silbisches R}: Wörter wie "prst" sind herausfordernd - das "r" fungiert als Vokal.
    \item \textbf{Lj und Nj}: Müssen als einzelne Laute palatalisiert werden, nicht als separate Buchstaben.
\end{enumerate}

\subsection{Übungspaare}

Diese Wortpaare helfen, ähnliche Laute zu unterscheiden:

\begin{center}
\begin{tabular}{llll}
\toprule
\textbf{Wort 1} & \textbf{Wort 2} & \textbf{Unterschied} & \textbf{Audio} \\
\midrule
\textbf{kuća} (Haus) & \textbf{kuča} (Hündin) & ć vs. č & \texttt{pairs/kuca\_kuca.mp3} \\
\textbf{đak} (Schüler) & \textbf{džak} (Sack) & đ vs. dž & \texttt{pairs/djak\_dzak.mp3} \\
\textbf{pas} (Hund) & \textbf{paš} (Pascha) & s vs. š & \texttt{pairs/pas\_pas.mp3} \\
\textbf{luk} (Zwiebel) & \textbf{ljuk} (Luke) & l vs. lj & \texttt{pairs/luk\_ljuk.mp3} \\
\textbf{zona} (Zone) & \textbf{žona} (Frau, regional) & z vs. ž & \texttt{pairs/zona\_zona.mp3} \\
\bottomrule
\end{tabular}
\end{center}

\section{Ausspracheübungen}

\subsection{Übung 1: Besondere Buchstaben}

Lies diese Wörter laut vor und überprüfe deine Aussprache mit den Audiodateien:

\begin{enumerate}
    \item čovjek - ptica - kuća - večer
    \item mačka - ćevapi - noć - ćup
    \item škola - šuma - miš - šećer
    \item žena - život - muž - nož
    \item džem - džep - džungla
    \item đak - rođendan - međed - đir
\end{enumerate}

Audio: \texttt{exercises/exercise1\_special\_letters.mp3}

\subsection{Übung 2: Minimalpaare}

Höre zu und wiederhole. Achte auf die Unterschiede:

\begin{enumerate}
    \item kuća (Haus) vs. kuča (Hündin)
    \item pas (Hund) vs. paš (Pascha)
    \item luk (Zwiebel) vs. ljuk (Luke)
    \item đak (Schüler) vs. džak (Sack)
    \item goniti (jagen) vs. gonjiti (vertreiben)
\end{enumerate}

Audio: \texttt{exercises/exercise2\_minimal\_pairs.mp3}

\subsection{Übung 3: Silbisches R}

Übe diese herausfordernden Wörter mit silbischem "r":

\begin{enumerate}
    \item prst (Finger)
    \item krv (Blut)
    \item vrh (Gipfel)
    \item smrt (Tod)
    \item trg (Platz)
    \item crv (Wurm)
    \item prvi (erster)
\end{enumerate}

Audio: \texttt{exercises/exercise3\_syllabic\_r.mp3}

\subsection{Übung 4: Zungenbrecher}

Diese Zungenbrecher helfen, schwierige Lautkombinationen zu üben:

\begin{enumerate}
    \item \textbf{Četiri crna čavka na četiri crna staba.}
    (Vier schwarze Dohlen auf vier schwarzen Stangen.)
    Audio: \texttt{exercises/tongue\_twister1.mp3}
    
    \item \textbf{Riba ribi grize rep.}
    (Fisch beißt Fisch in den Schwanz.)
    Audio: \texttt{exercises/tongue\_twister2.mp3}
    
    \item \textbf{Petar Petrić plete preko prsta.}
    (Peter Petrić strickt über seinen Finger.)
    Audio: \texttt{exercises/tongue\_twister3.mp3}
    
    \item \textbf{Šešir sa šeširom šiša šišmiša.}
    (Hut mit Hut schneidet Fledermaus die Haare.)
    Audio: \texttt{exercises/tongue\_twister4.mp3}
\end{enumerate}

\subsection{Übung 5: Vollständige Sätze}

Übe diese Sätze, die viele besondere Laute enthalten:

\begin{enumerate}
    \item Mačka sjedi u kući cijelu noć.
    (Die Katze sitzt die ganze Nacht im Haus.)
    
    \item Učitelj čita knjigu o ljudima i životinjama.
    (Der Lehrer liest ein Buch über Menschen und Tiere.)
    
    \item Žena kupuje šešir, čokoladu i džem.
    (Die Frau kauft einen Hut, Schokolade und Marmelade.)
    
    \item Đak ima rođendan u ljeto.
    (Der Schüler hat im Sommer Geburtstag.)
    
    \item Hrvatski jezik ima trideset slova.
    (Die kroatische Sprache hat dreißig Buchstaben.)
\end{enumerate}

Audio: \texttt{exercises/exercise5\_sentences.mp3}

\section{Tipps für deutsche Sprecher}

\begin{tcolorbox}[colback=lightblue!20, colframe=croatianblue, title=\textbf{Spezielle Tipps}]

\textbf{Was ist einfach für deutsche Sprecher:}
\begin{itemize}
    \item Š [ʃ] - genau wie deutsches "sch"
    \item C [ts] - genau wie deutsches "z"
    \item H [x] - genau wie deutsches "ch" in "Bach"
    \item Vokale ähneln deutschen reinen Vokalen
    \item Viele Wörter sind aus dem Deutschen entlehnt!
\end{itemize}

\textbf{Was Übung erfordert:}
\begin{itemize}
    \item Č von Ć unterscheiden (härter vs. weicher)
    \item Dž von Đ unterscheiden (härter vs. weicher)
    \item Das R konsequent rollen
    \item Silbisches R aussprechen (r als Vokal)
    \item Palatalisierte Lj- und Nj-Laute
    \item H am Ende von Wörtern nicht weglassen
\end{itemize}

\textbf{Lernstrategie:}
\begin{enumerate}
    \item Höre dir die Audiodateien wiederholt an
    \item Übe Minimalpaare (kuća vs. kuča)
    \item Nimm dich selbst auf und vergleiche mit Muttersprachlern
    \item Konzentriere dich auf die schwierigen Laute: ć, đ, gerolltes r
    \item Übe täglich Wörter mit silbischem r
    \item Schaue kroatische Medien mit Untertiteln
    \item Versuche, einen Sprachaustauschpartner zu finden
\end{enumerate}

\end{tcolorbox}

\section{Organisation des Audiomaterials}

Alle in diesem Leitfaden referenzierten Audiodateien sollten wie folgt organisiert werden:

\begin{verbatim}
audio/
├── pronunciation/
│   ├── letters/
│   │   ├── a.mp3, b.mp3, c.mp3, č.mp3, ć.mp3, ...
│   ├── words/
│   │   ├── cokolada.mp3, macka.mp3, skola.mp3, ...
│   ├── sentences/
│   │   ├── covjek_pije_caj.mp3, macka_spava.mp3, ...
│   ├── pairs/
│   │   ├── kuca_kuca.mp3, djak_dzak.mp3, ...
│   ├── exercises/
│   │   ├── exercise1_special_letters.mp3
│   │   ├── exercise2_minimal_pairs.mp3
│   │   ├── tongue_twister1.mp3
│   │   └── ...
\end{verbatim}

\textbf{Aufnahmehinweise für Audiodateien:}
\begin{itemize}
    \item Verwende einen kroatischen Muttersprachler (vorzugsweise Standard-Zagreber Dialekt)
    \item Aufnahme mit 44,1 kHz, 192 kbps MP3 oder höherer Qualität
    \item Deutlich und in moderatem Tempo für Lernende aussprechen
    \item Für Minimalpaare beide Wörter mit Pause dazwischen aufnehmen
    \item Für Sätze in normaler Sprechgeschwindigkeit aufnehmen
    \item 0,5 Sekunden Stille vor und nach jeder Aufnahme einfügen
\end{itemize}

\section{Schnellreferenztabelle}

\begin{center}
\begin{longtable}{lllll}
\caption{Kroatische Sonderbuchstaben - Schnellreferenz}\\
\toprule
\textbf{Buchstabe} & \textbf{IPA} & \textbf{Wie im Deutschen} & \textbf{Beispiel} & \textbf{Bedeutung} \\
\midrule
\endfirsthead
\multicolumn{5}{c}%
{\tablename\ \thetable\ -- Fortsetzung von vorheriger Seite} \\
\toprule
\textbf{Buchstabe} & \textbf{IPA} & \textbf{Wie im Deutschen} & \textbf{Beispiel} & \textbf{Bedeutung} \\
\midrule
\endhead
\midrule
\multicolumn{5}{r}{Fortsetzung auf nächster Seite...} \\
\endfoot
\bottomrule
\endlastfoot
Č, č & [tʃ] & tsch (Tschüss) & čokolada & Schokolade \\
Ć, ć & [tɕ] & weiches tsch & mačka & Katze \\
Š, š & [ʃ] & sch (Schule) & škola & Schule \\
Ž, ž & [ʒ] & stimmhaftes sch & žena & Frau \\
Dž, dž & [dʒ] & dsch (Dschungel) & džem & Marmelade \\
Đ, đ & [dʑ] & weiches dsch & đak & Schüler \\
C, c & [ts] & z (Zeit) & cura & Mädchen \\
Lj, lj & [ʎ] & palatalisiertes l & ljeto & Sommer \\
Nj, nj & [ɲ] & palatalisiertes n & konj & Pferd \\
R, r & [r] & gerolltes r & riba & Fisch \\
H, h & [x] & ch (Bach) & hrana & Essen \\
\end{longtable}
\end{center}

\section{Zusätzliche Ressourcen}

\textbf{Empfohlene Übung:}
\begin{itemize}
    \item Tägliche Ausspracheübung: 10-15 Minuten
    \item Konzentriere dich auf einen besonderen Laut pro Tag
    \item Verwende Audiomaterialien zusammen mit diesem Leitfaden
    \item Übe mit kroatischer Musik und Filmen
    \item Nimm dich selbst auf und vergleiche mit Muttersprachlern
\end{itemize}

\textbf{Online-Ressourcen:}
\begin{itemize}
    \item Forvo.com - Aussprache von Muttersprachlern
    \item YouTube kroatische Sprachkanäle
    \item Kroatisches Radio und TV (HRT)
    \item Sprachaustausch-Plattformen (Tandem, HelloTalk)
\end{itemize}

\vspace{1cm}

\begin{center}
\textbf{\Large Sretno s učenjem izgovora!}\\
\textbf{\Large Viel Erfolg beim Aussprachetraining!}
\end{center}


\chapter{Grammatik-Referenz}
% Grammar Reference
% This section provides a comprehensive overview of Croatian grammar covered in the course

\section{The Croatian Cases / Die kroatischen Fälle}

Croatian has 7 cases:
\begin{enumerate}
    \item Nominative (Nominativ) - Subject case
    \item Genitive (Genitiv) - Possession, negation
    \item Dative (Dativ) - Indirect object
    \item Accusative (Akkusativ) - Direct object
    \item Vocative (Vokativ) - Direct address
    \item Locative (Lokativ) - Location
    \item Instrumental (Instrumental) - Means, accompaniment
\end{enumerate}

\section{Verb Conjugation Tables / Verbkonjugationstabellen}

% To be developed with conjugation tables for common verbs

\section{Gender and Number / Geschlecht und Zahl}

% To be developed with rules for gender agreement

\section{Prepositions and Cases / Präpositionen und Fälle}

% To be developed with preposition usage

\section{Question Formation / Fragebildung}

% To be developed with question formation rules


\chapter{Wortschatzliste}
% Complete Vocabulary List
% Croatian-German vocabulary organized by units

\section{Unit 1: Welcome to Croatian}

\subsection*{The Alphabet / Das Alphabet}
\begin{itemize}
    \item abeceda - Alphabet
    \item slovo - Buchstabe
\end{itemize}

\subsection*{Greetings / Grüße}
\begin{itemize}
    \item bok / bog - hallo
    \item dobro jutro - guten Morgen
    \item dobar dan - guten Tag
    \item dobra večer - guten Abend
    \item laku noć - gute Nacht
    \item doviđenja - auf Wiedersehen
    \item ćao - tschüss
\end{itemize}

\subsection*{Polite Expressions / Höfliche Ausdrücke}
\begin{itemize}
    \item hvala - danke
    \item hvala lijepo - vielen Dank
    \item molim - bitte
    \item molim te - bitte (informal)
    \item molim vas - bitte (formal)
    \item oprosti / oprostite - entschuldigung
    \item izvini / izvinite - verzeihung
    \item nema problema - kein Problem
\end{itemize}

\subsection*{Basic Words / Grundwörter}
\begin{itemize}
    \item da - ja
    \item ne - nein
    \item u redu - in Ordnung, OK
    \item dobro - gut
    \item super - super
    \item odlično - ausgezeichnet
\end{itemize}

\subsection*{Self-Introduction / Selbstvorstellung}
\begin{itemize}
    \item ja - ich
    \item ja sam - ich bin
    \item zovem se - ich heiße
    \item imam ... godina - ich bin ... Jahre alt
    \item drago mi je - schön, dich/Sie kennenzulernen
    \item kako si? - wie geht es dir?
    \item kako ste? - wie geht es Ihnen?
\end{itemize}

\subsection*{Numbers 1-20 / Zahlen 1-20}
\begin{itemize}
    \item jedan - eins
    \item dva - zwei
    \item tri - drei
    \item četiri - vier
    \item pet - fünf
    \item šest - sechs
    \item sedam - sieben
    \item osam - acht
    \item devet - neun
    \item deset - zehn
    \item jedanaest - elf
    \item dvanaest - zwölf
    \item trinaest - dreizehn
    \item četrnaest - vierzehn
    \item petnaest - fünfzehn
    \item šesnaest - sechzehn
    \item sedamnaest - siebzehn
    \item osamnaest - achtzehn
    \item devetnaest - neunzehn
    \item dvadeset - zwanzig
\end{itemize}

\subsection*{Questions / Fragen}
\begin{itemize}
    \item koliko imaš godina? - wie alt bist du?
    \item kako se zoveš? - wie heißt du?
\end{itemize}

\subsection*{People / Menschen}
\begin{itemize}
    \item dijete / djeca - Kind / Kinder
    \item mama - Mama
    \item učitelj - Lehrer
\end{itemize}

\subsection*{Croatia Vocabulary / Kroatien-Wortschatz}
\begin{itemize}
    \item Hrvatska - Kroatien
    \item hrvatski - kroatisch / Kroatisch (Sprache)
    \item Zagreb - Zagreb
    \item glavni grad - Hauptstadt
    \item zastava - Flagge
    \item grb - Wappen
    \item šahovnica - Schachbrettmuster
    \item otok / otoci - Insel / Inseln
    \item more - Meer
    \item Jadransko more - Adriatisches Meer
    \item obala - Küste
    \item nacionalni park - Nationalpark
\end{itemize}

\subsection*{Colors (from flag) / Farben (von der Flagge)}
\begin{itemize}
    \item crvena - rot
    \item bijela - weiß
    \item plava - blau
\end{itemize}

\subsection*{Additional Vocabulary / Zusätzliches Vokabular}
\begin{itemize}
    \item čokolada - Schokolade
    \item mačka - Katze
    \item džem - Marmelade
    \item đak - Schüler
    \item škola - Schule
    \item žena - Frau
    \item ljubav - Liebe
    \item konj - Pferd
    \item danas - heute
\end{itemize}

\section{Unit 2: My Family and I}
% Vocabulary from Unit 2

\section{Unit 3: Colors and the World Around Us}
% Vocabulary from Unit 3

\section{Unit 4: School and Learning}
% Vocabulary from Unit 4

\section{Unit 5: My Home and My City}
% Vocabulary from Unit 5

\section{Unit 6: Food and Meals}
% Vocabulary from Unit 6

\section{Unit 7: Seasons and Weather}
% Vocabulary from Unit 7

\section{Unit 8: Hobbies and Free Time}
% Vocabulary from Unit 8

\section{Unit 9: Croatian History and Traditions}
% Vocabulary from Unit 9

\section{Unit 10: Holidays and Celebrations}
% Vocabulary from Unit 10

\section{Unit 11: Animals and Nature}
% Vocabulary from Unit 11

\section{Unit 12: Review}
% Additional vocabulary from Unit 12


\chapter{Lösungsschlüssel}
% Lösungsschlüssel für Übungen
% Lösungen für alle Übungen im Buch

\section{Einheit 1: Willkommen zu Kroatisch}

\subsection*{Übung 1: Alphabet-Übung}

A) Vollständiges kroatisches Alphabet:
A, B, C, Č, Ć, D, Dž, Đ, E, F, G, H, I, J, K, L, Lj, M, N, Nj, O, P, R, S, Š, T, U, V, Z, Ž

B) Die 8 besonderen Buchstaben sind: Č, Ć, Dž, Đ, Lj, Nj, Š, Ž

C) Zuordnung:
1. škola - b) \glqq Schule\grqq{}
2. čokolada - a) \glqq Schokolade\grqq{}
3. mačka - d) \glqq Katze\grqq{}
4. džem - c) \glqq Marmelade\grqq{}

\subsection*{Übung 2: Grüße}

A) Fülle die Lücken:
\begin{enumerate}
    \item Um 8:00 Uhr morgens: Dobro jutro!
    \item Um 14:00 Uhr: Dobar dan!
    \item Um 20:00 Uhr: Dobra večer!
    \item Vor dem Schlafengehen: Laku noć!
    \item Abschied: Doviđenja! (oder Bok!/Ćao!)
\end{enumerate}

B) Übersetzungen:
\begin{enumerate}
    \item Danke: Hvala!
    \item Bitte: Molim!
    \item Entschuldigung: Oprosti!/Oprostite! oder Izvini!/Izvinite!
    \item Kein Problem: Nema problema!
\end{enumerate}

\subsection*{Übung 3: Vorstellung}

Beispielantwort (variiert je nach Schüler):
Bok! Ja sam [Name]. Imam [Alter] godina. Dobro sam. Učim hrvatski jezik. Doviđenja!

\subsection*{Übung 4: Dialog-Übung}

Beispielantworten (variieren):
\begin{itemize}
    \item Bok! Ja sam [dein Name].
    \item Imam [dein Alter] godina. (A ti?)
    \item Dobro sam, hvala!
    \item Doviđenja! / Bok!
\end{itemize}

\subsection*{Übung 5: Zahlen}

A) Zahlen auf Kroatisch:
\begin{enumerate}
    \item 7: sedam
    \item 13: trinaest
    \item 18: osamnaest
    \item 5: pet
    \item 20: dvadeset
\end{enumerate}

B) Antworten variieren je nach Schüler. Beispiele:
\begin{itemize}
    \item Imam devet godina. (Ich bin neun Jahre alt)
    \item jedan, dva, tri, četiri, pet, šest, sedam, osam, devet, deset
    \item U mojoj školi ima dvadeset učenika. (In meiner Klasse sind 20 Schüler)
\end{itemize}

\subsection*{Übung 6: Kroatien-Fakten}

\begin{enumerate}
    \item Hauptstadt: Zagreb
    \item Inseln: 1.244 Inseln (nur 48 bewohnt)
    \item Flaggenfarben: Rot, Weiß und Blau (crvena, bijela, plava)
    \item Städte: (beliebige zwei) Zagreb, Split, Dubrovnik, Rijeka, Zadar, Pula
    \item Nationaltier: Baummarder (kuna)
    \item Krawatte: Kroatische Soldaten im 17. Jahrhundert
\end{enumerate}

\subsection*{Übung 7: Kreative Aktivität}

A) Zeichnung der kroatischen Flagge mit beschrifteten Farben: crvena (rot), bijela (weiß), plava (blau)

B) Poster variiert je nach Schüler

\subsection*{Übung 8: Hörübung}

Antworten variieren basierend auf Audio-Inhalten. Beispielantworten:
\begin{enumerate}
    \item Grüße: Bok, Dobro jutro, Dobar dan, Dobra večer, Laku noć, Doviđenja, Hvala, Molim
    \item Zahlen: (wie auf dem Track gehört)
    \item Namen: Luka, Ana, Ivan, Marta (aus den Dialogen)
\end{enumerate}

\subsection*{Zahlen-Übungsaktivitäten}

Aktivität 1: Zählspiel
\begin{enumerate}
    \item 5 Sterne: pet
    \item 7 Blumen: sedam
    \item 11 Äpfel: jedanaest
    \item 15 Bücher: petnaest
    \item 20 Katzen: dvadeset
\end{enumerate}

Aktivität 2: Mathematik auf Kroatisch
\begin{enumerate}
    \item pet + tri = osam (5 + 3 = 8)
    \item deset - dva = osam (10 - 2 = 8)
    \item sedam + šest = trinaest (7 + 6 = 13)
    \item petnaest - pet = deset (15 - 5 = 10)
    \item devet + jedan = deset (9 + 1 = 10)
\end{enumerate}

Aktivität 3: Telefonnummern
Antworten variieren je nach Schüler.

\section{Einheit 2: Meine Familie und ich}
% Antworten für Übungen aus Einheit 2

\section{Einheit 3: Farben und die Welt um uns herum}
% Antworten für Übungen aus Einheit 3

\section{Einheit 4: Schule und Lernen}
% Antworten für Übungen aus Einheit 4

\section{Einheit 5: Mein Zuhause und meine Stadt}
% Antworten für Übungen aus Einheit 5

\section{Einheit 6: Essen und Mahlzeiten}
% Antworten für Übungen aus Einheit 6

\section{Einheit 7: Jahreszeiten und Wetter}
% Antworten für Übungen aus Einheit 7

\section{Einheit 8: Hobbys und Freizeit}
% Antworten für Übungen aus Einheit 8

\section{Einheit 9: Kroatische Geschichte und Traditionen}
% Antworten für Übungen aus Einheit 9

\section{Einheit 10: Feiertage und Feste}
% Antworten für Übungen aus Einheit 10

\section{Einheit 11: Tiere und Natur}
% Antworten für Übungen aus Einheit 11

\section{Einheit 12: Wiederholung}
% Antworten für Übungen aus Einheit 12


\chapter{Kroatisch-Deutsch / Deutsch-Kroatisch Wörterbuch}
% Croatian-German / German-Croatian Dictionary
% Comprehensive dictionary of all vocabulary in the course

\section{Croatian-German / Kroatisch-Deutsch}

% A
% Alphabetically organized Croatian words with German translations

% B

% C

% Č

% Ć

% D

% (Continue through alphabet...)

\section{German-Croatian / Deutsch-Kroatisch}

% A
% Alphabetically organized German words with Croatian translations

% B

% C

% (Continue through alphabet...)


\end{document}
