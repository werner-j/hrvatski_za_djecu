% Lösungsschlüssel für Übungen
% Lösungen für alle Übungen im Buch

\section{Einheit 1: Willkommen zu Kroatisch}

\subsection*{Übung 1: Alphabet-Übung}

A) Vollständiges kroatisches Alphabet:
A, B, C, Č, Ć, D, Dž, Đ, E, F, G, H, I, J, K, L, Lj, M, N, Nj, O, P, R, S, Š, T, U, V, Z, Ž

B) Die 8 besonderen Buchstaben sind: Č, Ć, Dž, Đ, Lj, Nj, Š, Ž

C) Zuordnung:
1. škola - b) \glqq Schule\grqq{}
2. čokolada - a) \glqq Schokolade\grqq{}
3. mačka - d) \glqq Katze\grqq{}
4. džem - c) \glqq Marmelade\grqq{}

\subsection*{Übung 2: Grüße}

A) Fülle die Lücken:
\begin{enumerate}
    \item Um 8:00 Uhr morgens: Dobro jutro!
    \item Um 14:00 Uhr: Dobar dan!
    \item Um 20:00 Uhr: Dobra večer!
    \item Vor dem Schlafengehen: Laku noć!
    \item Abschied: Doviđenja! (oder Bok!/Ćao!)
\end{enumerate}

B) Übersetzungen:
\begin{enumerate}
    \item Danke: Hvala!
    \item Bitte: Molim!
    \item Entschuldigung: Oprosti!/Oprostite! oder Izvini!/Izvinite!
    \item Kein Problem: Nema problema!
\end{enumerate}

\subsection*{Übung 3: Vorstellung}

Beispielantwort (variiert je nach Schüler):
Bok! Ja sam [Name]. Imam [Alter] godina. Dobro sam. Učim hrvatski jezik. Doviđenja!

\subsection*{Übung 4: Dialog-Übung}

Beispielantworten (variieren):
\begin{itemize}
    \item Bok! Ja sam [dein Name].
    \item Imam [dein Alter] godina. (A ti?)
    \item Dobro sam, hvala!
    \item Doviđenja! / Bok!
\end{itemize}

\subsection*{Übung 5: Zahlen}

A) Zahlen auf Kroatisch:
\begin{enumerate}
    \item 7: sedam
    \item 13: trinaest
    \item 18: osamnaest
    \item 5: pet
    \item 20: dvadeset
\end{enumerate}

B) Antworten variieren je nach Schüler. Beispiele:
\begin{itemize}
    \item Imam devet godina. (Ich bin neun Jahre alt)
    \item jedan, dva, tri, četiri, pet, šest, sedam, osam, devet, deset
    \item U mojoj školi ima dvadeset učenika. (In meiner Klasse sind 20 Schüler)
\end{itemize}

\subsection*{Übung 6: Kroatien-Fakten}

\begin{enumerate}
    \item Hauptstadt: Zagreb
    \item Inseln: 1.244 Inseln (nur 48 bewohnt)
    \item Flaggenfarben: Rot, Weiß und Blau (crvena, bijela, plava)
    \item Städte: (beliebige zwei) Zagreb, Split, Dubrovnik, Rijeka, Zadar, Pula
    \item Nationaltier: Baummarder (kuna)
    \item Krawatte: Kroatische Soldaten im 17. Jahrhundert
\end{enumerate}

\subsection*{Übung 7: Kreative Aktivität}

A) Zeichnung der kroatischen Flagge mit beschrifteten Farben: crvena (rot), bijela (weiß), plava (blau)

B) Poster variiert je nach Schüler

\subsection*{Übung 8: Hörübung}

Antworten variieren basierend auf Audio-Inhalten. Beispielantworten:
\begin{enumerate}
    \item Grüße: Bok, Dobro jutro, Dobar dan, Dobra večer, Laku noć, Doviđenja, Hvala, Molim
    \item Zahlen: (wie auf dem Track gehört)
    \item Namen: Luka, Ana, Ivan, Marta (aus den Dialogen)
\end{enumerate}

\subsection*{Zahlen-Übungsaktivitäten}

Aktivität 1: Zählspiel
\begin{enumerate}
    \item 5 Sterne: pet
    \item 7 Blumen: sedam
    \item 11 Äpfel: jedanaest
    \item 15 Bücher: petnaest
    \item 20 Katzen: dvadeset
\end{enumerate}

Aktivität 2: Mathematik auf Kroatisch
\begin{enumerate}
    \item pet + tri = osam (5 + 3 = 8)
    \item deset - dva = osam (10 - 2 = 8)
    \item sedam + šest = trinaest (7 + 6 = 13)
    \item petnaest - pet = deset (15 - 5 = 10)
    \item devet + jedan = deset (9 + 1 = 10)
\end{enumerate}

Aktivität 3: Telefonnummern
Antworten variieren je nach Schüler.

\section{Einheit 2: Meine Familie und ich}
% Antworten für Übungen aus Einheit 2

\section{Einheit 3: Farben und die Welt um uns herum}
% Antworten für Übungen aus Einheit 3

\section{Einheit 4: Schule und Lernen}
% Antworten für Übungen aus Einheit 4

\section{Einheit 5: Mein Zuhause und meine Stadt}
% Antworten für Übungen aus Einheit 5

\section{Einheit 6: Essen und Mahlzeiten}
% Antworten für Übungen aus Einheit 6

\section{Einheit 7: Jahreszeiten und Wetter}
% Antworten für Übungen aus Einheit 7

\section{Einheit 8: Hobbys und Freizeit}
% Antworten für Übungen aus Einheit 8

\section{Einheit 9: Kroatische Geschichte und Traditionen}
% Antworten für Übungen aus Einheit 9

\section{Einheit 10: Feiertage und Feste}
% Antworten für Übungen aus Einheit 10

\section{Einheit 11: Tiere und Natur}
% Antworten für Übungen aus Einheit 11

\section{Einheit 12: Wiederholung}
% Antworten für Übungen aus Einheit 12
