% Vollständige Vokabelliste
% Kroatisch-Deutsches Vokabular organisiert nach Einheiten

\section{Einheit 1: Willkommen zu Kroatisch}

\subsection*{Das Alphabet}
\begin{itemize}
    \item abeceda - Alphabet
    \item slovo - Buchstabe
\end{itemize}

\subsection*{Grüße}
\begin{itemize}
    \item bok / bog - hallo
    \item dobro jutro - guten Morgen
    \item dobar dan - guten Tag
    \item dobra večer - guten Abend
    \item laku noć - gute Nacht
    \item doviđenja - auf Wiedersehen
    \item ćao - tschüss
\end{itemize}

\subsection*{Höfliche Ausdrücke}
\begin{itemize}
    \item hvala - danke
    \item hvala lijepo - vielen Dank
    \item molim - bitte
    \item molim te - bitte (informell)
    \item molim vas - bitte (formell)
    \item oprosti / oprostite - entschuldigung
    \item izvini / izvinite - verzeihung
    \item nema problema - kein Problem
\end{itemize}

\subsection*{Grundwörter}
\begin{itemize}
    \item da - ja
    \item ne - nein
    \item u redu - in Ordnung, OK
    \item dobro - gut
    \item super - super
    \item odlično - ausgezeichnet
\end{itemize}

\subsection*{Selbstvorstellung}
\begin{itemize}
    \item ja - ich
    \item ja sam - ich bin
    \item zovem se - ich heiße
    \item imam ... godina - ich bin ... Jahre alt
    \item drago mi je - schön, dich/Sie kennenzulernen
    \item kako si? - wie geht es dir?
    \item kako ste? - wie geht es Ihnen?
\end{itemize}

\subsection*{Zahlen 1-20}
\begin{itemize}
    \item jedan - eins
    \item dva - zwei
    \item tri - drei
    \item četiri - vier
    \item pet - fünf
    \item šest - sechs
    \item sedam - sieben
    \item osam - acht
    \item devet - neun
    \item deset - zehn
    \item jedanaest - elf
    \item dvanaest - zwölf
    \item trinaest - dreizehn
    \item četrnaest - vierzehn
    \item petnaest - fünfzehn
    \item šesnaest - sechzehn
    \item sedamnaest - siebzehn
    \item osamnaest - achtzehn
    \item devetnaest - neunzehn
    \item dvadeset - zwanzig
\end{itemize}

\subsection*{Fragen}
\begin{itemize}
    \item koliko imaš godina? - wie alt bist du?
    \item kako se zoveš? - wie heißt du?
\end{itemize}

\subsection*{Menschen}
\begin{itemize}
    \item dijete / djeca - Kind / Kinder
    \item mama - Mama
    \item učitelj - Lehrer
\end{itemize}

\subsection*{Kroatien-Wortschatz}
\begin{itemize}
    \item Hrvatska - Kroatien
    \item hrvatski - kroatisch / Kroatisch (Sprache)
    \item Zagreb - Zagreb
    \item glavni grad - Hauptstadt
    \item zastava - Flagge
    \item grb - Wappen
    \item šahovnica - Schachbrettmuster
    \item otok / otoci - Insel / Inseln
    \item more - Meer
    \item Jadransko more - Adriatisches Meer
    \item obala - Küste
    \item nacionalni park - Nationalpark
\end{itemize}

\subsection*{Farben (von der Flagge)}
\begin{itemize}
    \item crvena - rot
    \item bijela - weiß
    \item plava - blau
\end{itemize}

\subsection*{Zusätzliches Vokabular}
\begin{itemize}
    \item čokolada - Schokolade
    \item mačka - Katze
    \item džem - Marmelade
    \item đak - Schüler
    \item škola - Schule
    \item žena - Frau
    \item ljubav - Liebe
    \item konj - Pferd
    \item danas - heute
\end{itemize}

\section{Einheit 2: Meine Familie und ich}
% Vokabular aus Einheit 2

\section{Einheit 3: Farben und die Welt um uns herum}
% Vokabular aus Einheit 3

\section{Einheit 4: Schule und Lernen}
% Vokabular aus Einheit 4

\section{Einheit 5: Mein Zuhause und meine Stadt}
% Vokabular aus Einheit 5

\section{Einheit 6: Essen und Mahlzeiten}
% Vokabular aus Einheit 6

\section{Einheit 7: Jahreszeiten und Wetter}
% Vokabular aus Einheit 7

\section{Einheit 8: Hobbys und Freizeit}
% Vokabular aus Einheit 8

\section{Einheit 9: Kroatische Geschichte und Traditionen}
% Vokabular aus Einheit 9

\section{Einheit 10: Feiertage und Feste}
% Vokabular aus Einheit 10

\section{Einheit 11: Tiere und Natur}
% Vokabular aus Einheit 11

\section{Einheit 12: Wiederholung}
% Zusätzliches Vokabular aus Einheit 12
