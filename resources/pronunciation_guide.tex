% Pronunciation Guide for Croatian Language
% Comprehensive guide comparing Croatian sounds with German pronunciation

\section{Einführung}

Dieser Ausspracheführer bietet detaillierte Informationen über kroatische Laute, mit besonderem Fokus auf Buchstaben und Laute, die sich vom Deutschen unterscheiden. Jeder Laut enthält IPA-Transkription (Internationales Phonetisches Alphabet), Vergleich mit Deutsch, häufige Fehler und Übungswörter.

\subsection{Audio-Referenzen}

Audiodateien für alle Aussprachebeispiele finden Sie in den begleitenden digitalen Materialien:
\begin{itemize}
    \item \small\texttt{audio/pronunciation/letters/} - Einzelne Buchstabenlaute
    \item \small\texttt{audio/pronunciation/words/} - Übungswörter
    \item \small\texttt{audio/pronunciation/sentences/} - Beispielsätze
\end{itemize}

\section{Das kroatische Alphabet}

Das kroatische Alphabet (kroatisch: \textit{hrvatska abeceda} oder \textit{gajica}) hat 30 Buchstaben. Es verwendet die lateinische Schrift mit mehreren speziellen diakritischen Zeichen.

\subsection{Vollständiges Alphabet}

\begin{center}
\begin{tabular}{lll}
\toprule
\textbf{Buchstabe} & \textbf{IPA} & \textbf{Audio-Referenz} \\
\midrule
A, a & [a] & \small\texttt{audio/pronunciation/letters/a.mp3} \\
B, b & [b] & \small\texttt{audio/pronunciation/letters/b.mp3} \\
C, c & [ts] & \small\texttt{audio/pronunciation/letters/c.mp3} \\
Č, č & [tʃ] & \small\texttt{audio/pronunciation/letters/č.mp3} \\
Ć, ć & [tɕ] & \small\texttt{audio/pronunciation/letters/ć.mp3} \\
D, d & [d] & \small\texttt{audio/pronunciation/letters/d.mp3} \\
Dž, dž & [dʒ] & \small\texttt{audio/pronunciation/letters/dž.mp3} \\
Đ, đ & [dʑ] & \small\texttt{audio/pronunciation/letters/đ.mp3} \\
E, e & [e] & \small\texttt{audio/pronunciation/letters/e.mp3} \\
F, f & [f] & \small\texttt{audio/pronunciation/letters/f.mp3} \\
G, g & [ɡ] & \small\texttt{audio/pronunciation/letters/g.mp3} \\
H, h & [x] & \small\texttt{audio/pronunciation/letters/h.mp3} \\
I, i & [i] & \small\texttt{audio/pronunciation/letters/i.mp3} \\
J, j & [j] & \small\texttt{audio/pronunciation/letters/j.mp3} \\
K, k & [k] & \small\texttt{audio/pronunciation/letters/k.mp3} \\
L, l & [l] & \small\texttt{audio/pronunciation/letters/l.mp3} \\
Lj, lj & [ʎ] & \small\texttt{audio/pronunciation/letters/lj.mp3} \\
M, m & [m] & \small\texttt{audio/pronunciation/letters/m.mp3} \\
N, n & [n] & \small\texttt{audio/pronunciation/letters/n.mp3} \\
Nj, nj & [ɲ] & \small\texttt{audio/pronunciation/letters/nj.mp3} \\
O, o & [o] & \small\texttt{audio/pronunciation/letters/o.mp3} \\
P, p & [p] & \small\texttt{audio/pronunciation/letters/p.mp3} \\
R, r & [r] & \small\texttt{audio/pronunciation/letters/r.mp3} \\
S, s & [s] & \small\texttt{audio/pronunciation/letters/s.mp3} \\
Š, š & [ʃ] & \small\texttt{audio/pronunciation/letters/š.mp3} \\
T, t & [t] & \small\texttt{audio/pronunciation/letters/t.mp3} \\
U, u & [u] & \small\texttt{audio/pronunciation/letters/u.mp3} \\
V, v & [ʋ] & \small\texttt{audio/pronunciation/letters/v.mp3} \\
Z, z & [z] & \small\texttt{audio/pronunciation/letters/z.mp3} \\
Ž, ž & [ʒ] & \small\texttt{audio/pronunciation/letters/ž.mp3} \\
\bottomrule
\end{tabular}
\end{center}

\section{Besondere kroatische Buchstaben}

Diese Buchstaben gibt es im Deutschen nicht und erfordern besondere Aufmerksamkeit.

\subsection{Č, č [tʃ] - "Tsch-Laut"}

\begin{tcolorbox}[breakable, colback=lightblue!30, colframe=croatianblue, title=\textbf{Č, č}]

\textbf{IPA:} [tʃ]

\textbf{Deutscher Vergleich:}
Ähnlich wie deutsches "tsch" in "Tschüss" oder "Deutsch".

\textbf{Wie man ausspricht:}
Platziere deine Zunge hinter deinen oberen Zähnen und lasse Luft mit einem "tsch"-Laut entweichen.

\textbf{Häufige Fehler:}
\begin{itemize}
    \item Als einfachen "c" [ts]-Laut aussprechen
    \item Nicht stark genug machen (es sollte klar und deutlich sein)
    \item Mit "ć" verwechseln (das weicher ist)
\end{itemize}

\textbf{Übungswörter:}
\begin{center}
\small
\begin{tabular}{lll}
\textbf{Kroatisch} & \textbf{Deutsch} & \textbf{Audio} \\
\midrule
\textbf{č}okolada & Schokolade & \small\texttt{words/cokolada.mp3} \\
\textbf{č}aj & Tee & \small\texttt{words/caj.mp3} \\
u\textbf{č}itelj & Lehrer & \small\texttt{words/ucitelj.mp3} \\
\textbf{č}ovjek & Mann, Mensch & \small\texttt{words/covjek.mp3} \\
de\textbf{č}ko & Junge & \small\texttt{words/decko.mp3} \\
\end{tabular}
\end{center}

\textbf{Übungssatz:}
\textit{Čovjek pije čaj i jede čokoladu.}
(Der Mann trinkt Tee und isst Schokolade.)
Audio: \small\texttt{sentences/covjek\_pije\_caj.mp3}

\end{tcolorbox}

\subsection{Ć, ć [tɕ] - "Weiches Tsch"}

\begin{tcolorbox}[breakable, colback=lightblue!30, colframe=croatianblue, title=\textbf{Ć, ć}]

\textbf{IPA:} [tɕ]

\textbf{Deutscher Vergleich:}
Ähnlich wie "č", aber weicher und palataler. Wie "tsch" sagen, während man lächelt oder mit der Zunge weiter vorne. Es gibt keine exakte deutsche Entsprechung.

\textbf{Wie man ausspricht:}
Beginne wie "č", aber platziere deine Zunge höher und weiter vorne zum harten Gaumen. Es ist ein "weicherer" Laut.

\textbf{Häufige Fehler:}
\begin{itemize}
    \item Genau wie "č" aussprechen (der häufigste Fehler!)
    \item Zu hart/rau machen
    \item Nicht zwischen "ć" und "č" unterscheiden (sie sind verschiedene Phoneme!)
\end{itemize}

\textbf{Übungswörter:}
\begin{center}
\small
\begin{tabular}{lll}
\textbf{Kroatisch} & \textbf{Deutsch} & \textbf{Audio} \\
\midrule
ma\textbf{ć}ka & Katze & \small\texttt{words/macka.mp3} \\
\textbf{ć}up & Krug & \small\texttt{words/cup.mp3} \\
no\textbf{ć} & Nacht & \small\texttt{words/noc.mp3} \\
dje\textbf{ć}a & Kinder & \small\texttt{words/djeca.mp3} \\
\textbf{ć}ao & Tschüss (informell) & \small\texttt{words/cao.mp3} \\
\end{tabular}
\end{center}

\textbf{Übungssatz:}
\textit{Mačka spava cijelu noć.}
(Die Katze schläft die ganze Nacht.)
Audio: \small\texttt{sentences/macka\_spava.mp3}

\textbf{Wichtiger Hinweis:}
Der Unterschied zwischen "č" und "ć" ist im Kroatischen entscheidend! Vergleiche:
\begin{itemize}
    \item \textbf{kuća} (Haus) vs. \textbf{kuča} (Hündin) - verschiedene Bedeutungen!
\end{itemize}

\end{tcolorbox}

\subsection{Š, š [ʃ] - "Sch-Laut"}

\begin{tcolorbox}[breakable, colback=lightblue!30, colframe=croatianblue, title=\textbf{Š, š}]

\textbf{IPA:} [ʃ]

\textbf{Deutscher Vergleich:}
Genau wie deutsches "sch" in "Schule" oder "schön".

\textbf{Wie man ausspricht:}
Das ist einfach für deutsche Sprecher! Sprich es genau wie deutsches "sch" aus.

\textbf{Häufige Fehler:}
\begin{itemize}
    \item Sehr wenige Fehler für deutsche Sprecher!
    \item Manchmal mit "s" in der Schrift verwechselt
\end{itemize}

\textbf{Übungswörter:}
\begin{center}
\small
\begin{tabular}{lll}
\textbf{Croatian} & \textbf{German} & \textbf{Audio} \\
\midrule
\textbf{š}kola & Schule & \small\texttt{words/skola.mp3} \\
\textbf{š}uma & Wald & \small\texttt{words/suma.mp3} \\
mi\textbf{š} & Maus & \small\texttt{words/mis.mp3} \\
kru\textbf{š}ka & Birne & \small\texttt{words/kruska.mp3} \\
\textbf{š}ecer & Zucker & \small\texttt{words/secer.mp3} \\
\end{tabular}
\end{center}

\textbf{Übungssatz:}
\textit{Škola ima veliku šumu.}
(Die Schule hat einen großen Wald.)
Audio: \small\texttt{sentences/skola\_ima\_sumu.mp3}

\end{tcolorbox}

\subsection{Ž, ž [ʒ] - "Stimmhaftes Sch"}

\begin{tcolorbox}[breakable, colback=lightblue!30, colframe=croatianblue, title=\textbf{Ž, ž}]

\textbf{IPA:} [ʒ]

\textbf{Deutscher Vergleich:}
Wie französisches "j" in "journal". Ähnlich wie deutsches "sch", aber stimmhaft (die Stimmbänder vibrieren).

\textbf{Wie man ausspricht:}
Beginne mit "š" [ʃ] und füge Stimme hinzu (bringe deine Stimmbänder zum Vibrieren). Lege deine Hand auf deinen Hals - du solltest Vibrationen spüren.

\textbf{Häufige Fehler:}
\begin{itemize}
    \item Als stimmloses "š" statt stimmhaftes "ž" aussprechen
    \item Mit "z" [z] verwechseln
    \item Nicht deutlich genug von "š" unterscheiden
\end{itemize}

\textbf{Übungswörter:}
\begin{center}
\small
\begin{tabular}{lll}
\textbf{Croatian} & \textbf{German} & \textbf{Audio} \\
\midrule
\textbf{ž}ena & Frau & \small\texttt{words/zena.mp3} \\
\textbf{ž}aba & Frosch & \small\texttt{words/zaba.mp3} \\
no\textbf{ž} & Messer & \small\texttt{words/noz.mp3} \\
ko\textbf{ž}a & Haut, Leder & \small\texttt{words/koza.mp3} \\
mu\textbf{ž} & Ehemann & \small\texttt{words/muz.mp3} \\
\end{tabular}
\end{center}

\textbf{Übungssatz:}
\textit{Žena gleda žabu u šumi.}
(Die Frau beobachtet einen Frosch im Wald.)
Audio: \small\texttt{sentences/zena\_gleda\_zabu.mp3}

\end{tcolorbox}

\subsection{Dž, dž [dʒ] - "Dsch-Laut"}

\begin{tcolorbox}[breakable, colback=lightblue!30, colframe=croatianblue, title=\textbf{Dž, dž}]

\textbf{IPA:} [dʒ]

\textbf{Deutscher Vergleich:}
Wie "dsch" im deutschen Lehnwort "Dschungel" oder wie das "j" in italienischen/spanischen Wörtern (z.B. "Giovanni"). Ähnlich wie die stimmhafte Version von "č".

\textbf{Wie man ausspricht:}
Beginne mit "č" [tʃ] und füge Stimme hinzu (vibriere die Stimmbänder). Es ist das stimmhafte Gegenstück zu "č".

\textbf{Häufige Fehler:}
\begin{itemize}
    \item Als "č" ohne Stimmhaftigkeit aussprechen
    \item In zwei separate Laute "d" + "ž" aufteilen
    \item Dies ist ein weniger häufiger Laut im Kroatischen, daher üben Schüler ihn möglicherweise nicht genug
\end{itemize}

\textbf{Übungswörter:}
\begin{center}
\small
\begin{tabular}{lll}
\textbf{Croatian} & \textbf{German} & \textbf{Audio} \\
\midrule
\textbf{dž}em & Marmelade & \small\texttt{words/dzem.mp3} \\
\textbf{dž}ep & Tasche & \small\texttt{words/dzep.mp3} \\
\textbf{dž}ungle & Dschungel & \small\texttt{words/dzungle.mp3} \\
\textbf{dž}ez & Jazz & \small\texttt{words/dzez.mp3} \\
\textbf{dž}in & Gin & \small\texttt{words/dzin.mp3} \\
\end{tabular}
\end{center}

\textbf{Übungssatz:}
\textit{Džem je u džepu.}
(Die Marmelade ist in der Tasche.)
Audio: \small\texttt{sentences/dzem\_u\_dzepu.mp3}

\textbf{Hinweis:}
Viele Wörter mit "dž" sind aus anderen Sprachen entlehnt.

\end{tcolorbox}

\subsection{Đ, đ [dʑ] - "Weiches Dsch"}

\begin{tcolorbox}[breakable, colback=lightblue!30, colframe=croatianblue, title=\textbf{Đ, đ}]

\textbf{IPA:} [dʑ]

\textbf{Deutscher Vergleich:}
Die stimmhafte (mit Stimmbandvibration) und weichere Version von "ć". Ähnlich wie der "d"-Laut im italienischen "Giovanni". Keine exakte deutsche Entsprechung.

\textbf{Wie man ausspricht:}
Dies ist der schwerste Laut für deutsche Sprecher! Er ist wie "dž", aber weicher und palataler (Zunge nach vorne). Denke daran als die stimmhafte Version von "ć".

\textbf{Häufige Fehler:}
\begin{itemize}
    \item Genau wie "dž" aussprechen (häufigster Fehler!)
    \item Nicht weich/palatal genug machen
    \item Mit "j" oder einfachem "d" verwechseln
    \item Nicht zwischen "đ" und "dž" unterscheiden
\end{itemize}

\textbf{Übungswörter:}
\begin{center}
\small
\begin{tabular}{lll}
\textbf{Croatian} & \textbf{German} & \textbf{Audio} \\
\midrule
\textbf{đ}ak & Schüler & \small\texttt{words/djak.mp3} \\
me\textbf{đ}ed & Bär & \small\texttt{words/medjed.mp3} \\
ro\textbf{đ}endan & Geburtstag & \small\texttt{words/rodjendan.mp3} \\
\textbf{đ}ir & Ingwer & \small\texttt{words/djir.mp3} \\
\textbf{đ}ubre & Dünger & \small\texttt{words/djubre.mp3} \\
\end{tabular}
\end{center}

\textbf{Übungssatz:}
\textit{Đak ima rođendan.}
(Der Schüler hat Geburtstag.)
Audio: \small\texttt{sentences/djak\_rodjendan.mp3}

\textbf{Wichtiger Hinweis:}
Wie bei "ć" vs. "č" ist der Unterschied zwischen "đ" und "dž" wichtig:
\begin{itemize}
    \item \textbf{đak} (Schüler) vs. \textbf{džak} (Sack) - verschiedene Bedeutungen!
\end{itemize}

\end{tcolorbox}

\subsection{C, c [ts] - "Z-Laut"}

\begin{tcolorbox}[breakable, colback=lightyellow!30, colframe=orange, title=\textbf{C, c}]

\textbf{IPA:} [ts]

\textbf{Deutscher Vergleich:}
Genau wie deutsches "z" in "Zeit" oder "Katze".

\textbf{Wie man ausspricht:}
Das ist einfach für deutsche Sprecher! Sprich es genau wie deutsches "z" aus.

\textbf{Häufige Fehler:}
\begin{itemize}
    \item Als "k" oder "s" aussprechen
    \item Schreibverwirrung: Kroatisches "c" = deutscher "z"-Laut
\end{itemize}

\textbf{Übungswörter:}
\begin{center}
\small
\begin{tabular}{lll}
\textbf{Croatian} & \textbf{German} & \textbf{Audio} \\
\midrule
\textbf{c}ura & Mädchen & \small\texttt{words/cura.mp3} \\
\textbf{c}vijet & Blume & \small\texttt{words/cvijet.mp3} \\
o\textbf{c}a & Vater & \small\texttt{words/oca.mp3} \\
u\textbf{c}a & Onkel & \small\texttt{words/uca.mp3} \\
me\textbf{c} & Bär & \small\texttt{words/mec.mp3} \\
\end{tabular}
\end{center}

\textbf{Übungssatz:}
\textit{Cura nosi cvijet.}
(Das Mädchen trägt eine Blume.)
Audio: \small\texttt{sentences/cura\_cvijet.mp3}

\end{tcolorbox}

\subsection{Lj, lj [ʎ] - "Palatales L"}

\begin{tcolorbox}[breakable, colback=lightgreen!30, colframe=green!60!black, title=\textbf{Lj, lj}]

\textbf{IPA:} [ʎ]

\textbf{Deutscher Vergleich:}
Ähnlich wie italienisches "gli" in "famiglia" oder spanisches "ll" in "llamar". Wie "l" und "j" sehr schnell zusammen sagen. Keine exakte deutsche Entsprechung.

\textbf{Wie man ausspricht:}
Platziere deine Zunge, als würdest du "l" sagen, aber drücke die Mitte deiner Zunge gegen deinen harten Gaumen. Es ist ein palatalisiertes "l".

\textbf{Häufige Fehler:}
\begin{itemize}
    \item Als zwei separate Laute "l" + "j" aussprechen
    \item Wie ein einfaches "l" klingen lassen
    \item Nicht genug palatalisieren
\end{itemize}

\textbf{Übungswörter:}
\begin{center}
\small
\begin{tabular}{lll}
\textbf{Croatian} & \textbf{German} & \textbf{Audio} \\
\midrule
\textbf{lj}eto & Sommer & \small\texttt{words/ljeto.mp3} \\
zem\textbf{lj}a & Erde, Land & \small\texttt{words/zemlja.mp3} \\
\textbf{lj}ubav & Liebe & \small\texttt{words/ljubav.mp3} \\
\textbf{lj}uljati & schaukeln & \small\texttt{words/ljuljati.mp3} \\
\textbf{lj}udi & Menschen & \small\texttt{words/ljudi.mp3} \\
\end{tabular}
\end{center}

\textbf{Übungssatz:}
\textit{Ljudi vole ljeto i ljubav.}
(Menschen lieben Sommer und Liebe.)
Audio: \small\texttt{sentences/ljudi\_ljeto.mp3}

\end{tcolorbox}

\subsection{Nj, nj [ɲ] - "Palatales N"}

\begin{tcolorbox}[breakable, colback=lightgreen!30, colframe=green!60!black, title=\textbf{Nj, nj}]

\textbf{IPA:} [ɲ]

\textbf{Deutscher Vergleich:}
Ähnlich wie spanisches "ñ" in "señor" oder italienisches "gn" in "gnocchi". Wie französisches "gn" in "cognac". Ähnlich wie "canyon" schnell sagen.

\textbf{Wie man ausspricht:}
Beginne mit "n", aber drücke die Mitte deiner Zunge gegen deinen harten Gaumen. Es ist ein palatalisiertes "n".

\textbf{Häufige Fehler:}
\begin{itemize}
    \item Als zwei separate Laute "n" + "j" aussprechen
    \item Wie ein einfaches "n" klingen lassen
    \item Nicht genug palatalisieren
\end{itemize}

\textbf{Übungswörter:}
\begin{center}
\small
\begin{tabular}{lll}
\textbf{Croatian} & \textbf{German} & \textbf{Audio} \\
\midrule
ko\textbf{nj} & Pferd & \small\texttt{words/konj.mp3} \\
k\textbf{nj}iga & Buch & \small\texttt{words/knjiga.mp3} \\
pje\textbf{sm}a & Lied & \small\texttt{words/pjesma.mp3} \\
sje\textbf{nj}a & Schatten & \small\texttt{words/sjena.mp3} \\
\textbf{nj}emačk\textbf{i} & deutsch & \small\texttt{words/njemacki.mp3} \\
\end{tabular}
\end{center}

\textbf{Übungssatz:}
\textit{Konj čita knjigu.}
(Das Pferd liest ein Buch.) - Albern, aber einprägsam!
Audio: \small\texttt{sentences/konj\_knjiga.mp3}

\end{tcolorbox}

\section{Andere wichtige Laute}

\subsection{R, r [r] - "Gerolltes R"}

\begin{tcolorbox}[breakable, colback=lightyellow!30, colframe=orange, title=\textbf{R, r}]

\textbf{IPA:} [r]

\textbf{Deutscher Vergleich:}
Das kroatische "r" ist immer gerollt, wie das "r" in einigen deutschen Dialekten (Bayrisch) oder wie italienisches/spanisches "r". Anders als das Standard-deutsche "r", das guttural ist.

\textbf{Wie man ausspricht:}
Tippe oder rolle deine Zunge gegen den Zahndamm (direkt hinter den oberen Zähnen). Lass deine Zunge vibrieren.

\textbf{Häufige Fehler:}
\begin{itemize}
    \item Deutsches gutturales "r" statt gerolltes "r" verwenden
    \item Nicht genug rollen
    \item Schwierigkeiten mit silbischem "r" (siehe besonderer Hinweis unten)
\end{itemize}

\textbf{Übungswörter:}
\begin{center}
\small
\begin{tabular}{lll}
\textbf{Croatian} & \textbf{German} & \textbf{Audio} \\
\midrule
\textbf{r}iba & Fisch & \small\texttt{words/riba.mp3} \\
\textbf{r}uka & Hand & \small\texttt{words/ruka.mp3} \\
k\textbf{r}uh & Brot & \small\texttt{words/kruh.mp3} \\
p\textbf{r}ijatelj & Freund & \small\texttt{words/prijatelj.mp3} \\
\textbf{r}adost & Freude & \small\texttt{words/radost.mp3} \\
\end{tabular}
\end{center}

\textbf{Besonderer Hinweis - Silbisches R:}
Im Kroatischen kann "r" eine Silbe für sich bilden, ohne Vokal! Das ist sehr ungewöhnlich.

Beispiele:
\begin{itemize}
    \item \textbf{prst} (Finger) - ausgesprochen "prrst" mit gerolltem r als Vokal
    \item \textbf{krv} (Blut) - ausgesprochen "krrv"
    \item \textbf{vrh} (Spitze, Gipfel) - ausgesprochen "vrrh"
    \item \textbf{smrt} (Tod) - ausgesprochen "smrrt"
\end{itemize}

Audio: \small\texttt{words/syllabic\_r\_examples.mp3}

\end{tcolorbox}

\subsection{H, h [x] - "Ch-Laut"}

\begin{tcolorbox}[breakable, colback=lightyellow!30, colframe=orange, title=\textbf{H, h}]

\textbf{IPA:} [x]

\textbf{Deutscher Vergleich:}
Wie deutsches "ch" in "Bach" oder "noch" (nach hinteren Vokalen). NICHT wie das "h" in "Haus"!

\textbf{Wie man ausspricht:}
Produziere den Laut aus dem hinteren Teil deines Rachens, wie deutsches "ch" in "ach". Das ist einfach für deutsche Sprecher!

\textbf{Häufige Fehler:}
\begin{itemize}
    \item Als schwaches "h" aussprechen
    \item Nicht guttural genug machen
    \item In manchen Regionen wird "h" kaum ausgesprochen oder weggelassen - versuche, es klar auszusprechen
\end{itemize}

\textbf{Übungswörter:}
\begin{center}
\small
\begin{tabular}{lll}
\textbf{Croatian} & \textbf{German} & \textbf{Audio} \\
\midrule
\textbf{h}rana & Essen & \small\texttt{words/hrana.mp3} \\
\textbf{h}vaditi & fangen & \small\texttt{words/hvaditi.mp3} \\
\textbf{h}rvatski & kroatisch & \small\texttt{words/hrvatski.mp3} \\
du\textbf{h} & Geist & \small\texttt{words/duh.mp3} \\
sni\textbf{jeh} & Schnee & \small\texttt{words/snijeh.mp3} \\
\end{tabular}
\end{center}

\textbf{Übungssatz:}
\textit{Hrvatski jezik je lijep.}
(Die kroatische Sprache ist schön.)
Audio: \small\texttt{sentences/hrvatski\_jezik.mp3}

\end{tcolorbox}

\section{Vokale}

Das Kroatische hat 5 Vokale: A, E, I, O, U. Sie werden konstanter ausgesprochen als im Deutschen und ähneln italienischen oder spanischen Vokalen.

\begin{tcolorbox}[breakable, colback=lightgreen!30, colframe=green!60!black, title=\textbf{Kroatische Vokale}]

\begin{center}
\begin{tabular}{lll}
\toprule
\textbf{Buchstabe} & \textbf{IPA} & \textbf{Deutscher Vergleich} \\
\midrule
A, a & [a] & Wie "a" in "Mann" \\
E, e & [e] & Wie "ä" in "Käse" oder "e" in "Meer" \\
I, i & [i] & Wie "i" in "Lied" \\
O, o & [o] & Wie "o" in "groß" \\
U, u & [u] & Wie "u" in "gut" \\
\bottomrule
\end{tabular}
\end{center}

\textbf{Wichtige Hinweise:}
\begin{itemize}
    \item Kroatische Vokale werden immer gleich ausgesprochen (keine Variationen)
    \item Sie sind reine Vokale (Monophthonge), keine Diphthonge
    \item Sie werden in unbetonten Silben nicht reduziert (anders als im Deutschen)
\end{itemize}

\textbf{Übungswörter für Vokale:}
\begin{itemize}
    \item \textbf{mama} [mama] - Mama
    \item \textbf{bebe} [bebe] - Baby (informal)
    \item \textbf{kino} [kino] - Kino
    \item \textbf{lopta} [lopta] - Ball
    \item \textbf{ruka} [ruka] - Hand
\end{itemize}

Audio: \small\texttt{words/vowel\_examples.mp3}

\end{tcolorbox}

\section{Betonung und Akzent}

\begin{tcolorbox}[breakable, colback=white, colframe=croatianred, title=\textbf{Wortbetonung}]

Das Kroatische hat ein komplexes System von Tönen und Akzenten, aber für Anfänger sind hier die Grundregeln:

\textbf{Grundregeln:}
\begin{itemize}
    \item Die Betonung liegt nie auf der letzten Silbe (mit seltenen Ausnahmen)
    \item In den meisten Wörtern liegt die Betonung auf der ersten oder zweiten Silbe
    \item Es gibt kurze und lange Vokale (aber weniger ausgeprägt als im Deutschen)
\end{itemize}

\textbf{Beispiele:}
\begin{itemize}
    \item \textbf{KU}ća (Haus) - Betonung auf der ersten Silbe
    \item \textbf{ŠKO}la (Schule) - Betonung auf der ersten Silbe
    \item pri\textbf{JA}telj (Freund) - Betonung auf der zweiten Silbe
\end{itemize}

\textbf{Für Anfänger:}
Mache dir anfangs nicht zu viele Gedanken über das komplexe Tonsystem. Konzentriere dich auf:
\begin{enumerate}
    \item Niemals die letzte Silbe betonen
    \item Muttersprachlern zuhören
    \item Mit Audiomaterialien üben
\end{enumerate}

\end{tcolorbox}

\section{Häufige Ausspracheprobleme}

\subsection{Zusammenfassung häufiger Fehler}

\begin{enumerate}
    \item \textbf{Č vs. Ć}: Die schwierigste Unterscheidung! Ć ist weicher und weiter vorne.
    \item \textbf{Dž vs. Đ}: Ähnlich ist đ weicher als dž.
    \item \textbf{Gerolltes R}: Deutsche Sprecher müssen lernen, das "r" zu rollen.
    \item \textbf{H-Laut}: Muss wie deutsches "ch" in "Bach" ausgesprochen werden, nicht weggelassen.
    \item \textbf{C-Laut}: Denke daran, dass es immer [ts] ist wie deutsches "z", niemals [k] oder [s].
    \item \textbf{Silbisches R}: Wörter wie "prst" sind herausfordernd - das "r" fungiert als Vokal.
    \item \textbf{Lj und Nj}: Müssen als einzelne Laute palatalisiert werden, nicht als separate Buchstaben.
\end{enumerate}

\subsection{Übungspaare}

Diese Wortpaare helfen, ähnliche Laute zu unterscheiden:

\begin{center}
\begin{tabular}{llll}
\toprule
\textbf{Wort 1} & \textbf{Wort 2} & \textbf{Unterschied} & \textbf{Audio} \\
\midrule
\textbf{kuća} (Haus) & \textbf{kuča} (Hündin) & ć vs. č & \small\texttt{pairs/kuca\_kuca.mp3} \\
\textbf{đak} (Schüler) & \textbf{džak} (Sack) & đ vs. dž & \small\texttt{pairs/djak\_dzak.mp3} \\
\textbf{pas} (Hund) & \textbf{paš} (Pascha) & s vs. š & \small\texttt{pairs/pas\_pas.mp3} \\
\textbf{luk} (Zwiebel) & \textbf{ljuk} (Luke) & l vs. lj & \small\texttt{pairs/luk\_ljuk.mp3} \\
\textbf{zona} (Zone) & \textbf{žona} (Frau, regional) & z vs. ž & \small\texttt{pairs/zona\_zona.mp3} \\
\bottomrule
\end{tabular}
\end{center}

\section{Ausspracheübungen}

\subsection{Übung 1: Besondere Buchstaben}

Lies diese Wörter laut vor und überprüfe deine Aussprache mit den Audiodateien:

\begin{enumerate}
    \item čovjek - ptica - kuća - večer
    \item mačka - ćevapi - noć - ćup
    \item škola - šuma - miš - šećer
    \item žena - život - muž - nož
    \item džem - džep - džungla
    \item đak - rođendan - međed - đir
\end{enumerate}

Audio: \small\texttt{exercises/exercise1\_special\_letters.mp3}

\subsection{Übung 2: Minimalpaare}

Höre zu und wiederhole. Achte auf die Unterschiede:

\begin{enumerate}
    \item kuća (Haus) vs. kuča (Hündin)
    \item pas (Hund) vs. paš (Pascha)
    \item luk (Zwiebel) vs. ljuk (Luke)
    \item đak (Schüler) vs. džak (Sack)
    \item goniti (jagen) vs. gonjiti (vertreiben)
\end{enumerate}

Audio: \small\texttt{exercises/exercise2\_minimal\_pairs.mp3}

\subsection{Übung 3: Silbisches R}

Übe diese herausfordernden Wörter mit silbischem "r":

\begin{enumerate}
    \item prst (Finger)
    \item krv (Blut)
    \item vrh (Gipfel)
    \item smrt (Tod)
    \item trg (Platz)
    \item crv (Wurm)
    \item prvi (erster)
\end{enumerate}

Audio: \small\texttt{exercises/exercise3\_syllabic\_r.mp3}

\subsection{Übung 4: Zungenbrecher}

Diese Zungenbrecher helfen, schwierige Lautkombinationen zu üben:

\begin{enumerate}
    \item \textbf{Četiri crna čavka na četiri crna staba.}
    (Vier schwarze Dohlen auf vier schwarzen Stangen.)
    Audio: \small\texttt{exercises/tongue\_twister1.mp3}
    
    \item \textbf{Riba ribi grize rep.}
    (Fisch beißt Fisch in den Schwanz.)
    Audio: \small\texttt{exercises/tongue\_twister2.mp3}
    
    \item \textbf{Petar Petrić plete preko prsta.}
    (Peter Petrić strickt über seinen Finger.)
    Audio: \small\texttt{exercises/tongue\_twister3.mp3}
    
    \item \textbf{Šešir sa šeširom šiša šišmiša.}
    (Hut mit Hut schneidet Fledermaus die Haare.)
    Audio: \small\texttt{exercises/tongue\_twister4.mp3}
\end{enumerate}

\subsection{Übung 5: Vollständige Sätze}

Übe diese Sätze, die viele besondere Laute enthalten:

\begin{enumerate}
    \item Mačka sjedi u kući cijelu noć.
    (Die Katze sitzt die ganze Nacht im Haus.)
    
    \item Učitelj čita knjigu o ljudima i životinjama.
    (Der Lehrer liest ein Buch über Menschen und Tiere.)
    
    \item Žena kupuje šešir, čokoladu i džem.
    (Die Frau kauft einen Hut, Schokolade und Marmelade.)
    
    \item Đak ima rođendan u ljeto.
    (Der Schüler hat im Sommer Geburtstag.)
    
    \item Hrvatski jezik ima trideset slova.
    (Die kroatische Sprache hat dreißig Buchstaben.)
\end{enumerate}

Audio: \small\texttt{exercises/exercise5\_sentences.mp3}

\section{Tipps für deutsche Sprecher}

\begin{tcolorbox}[breakable, colback=lightblue!20, colframe=croatianblue, title=\textbf{Spezielle Tipps}]

\textbf{Was ist einfach für deutsche Sprecher:}
\begin{itemize}
    \item Š [ʃ] - genau wie deutsches "sch"
    \item C [ts] - genau wie deutsches "z"
    \item H [x] - genau wie deutsches "ch" in "Bach"
    \item Vokale ähneln deutschen reinen Vokalen
    \item Viele Wörter sind aus dem Deutschen entlehnt!
\end{itemize}

\textbf{Was Übung erfordert:}
\begin{itemize}
    \item Č von Ć unterscheiden (härter vs. weicher)
    \item Dž von Đ unterscheiden (härter vs. weicher)
    \item Das R konsequent rollen
    \item Silbisches R aussprechen (r als Vokal)
    \item Palatalisierte Lj- und Nj-Laute
    \item H am Ende von Wörtern nicht weglassen
\end{itemize}

\textbf{Lernstrategie:}
\begin{enumerate}
    \item Höre dir die Audiodateien wiederholt an
    \item Übe Minimalpaare (kuća vs. kuča)
    \item Nimm dich selbst auf und vergleiche mit Muttersprachlern
    \item Konzentriere dich auf die schwierigen Laute: ć, đ, gerolltes r
    \item Übe täglich Wörter mit silbischem r
    \item Schaue kroatische Medien mit Untertiteln
    \item Versuche, einen Sprachaustauschpartner zu finden
\end{enumerate}

\end{tcolorbox}

\section{Organisation des Audiomaterials}

Alle in diesem Leitfaden referenzierten Audiodateien sollten wie folgt organisiert werden:

\begin{verbatim}
audio/
|-- pronunciation/
|   |-- letters/
|   |   |-- a.mp3, b.mp3, c.mp3, c.mp3, c.mp3, ...
|   |-- words/
|   |   |-- cokolada.mp3, macka.mp3, skola.mp3, ...
|   |-- sentences/
|   |   |-- covjek_pije_caj.mp3, macka_spava.mp3, ...
|   |-- pairs/
|   |   |-- kuca_kuca.mp3, djak_dzak.mp3, ...
|   |-- exercises/
|   |   |-- exercise1_special_letters.mp3
|   |   |-- exercise2_minimal_pairs.mp3
|   |   |-- tongue_twister1.mp3
|   |   |-- ...
\end{verbatim}

\textbf{Aufnahmehinweise für Audiodateien:}
\begin{itemize}
    \item Verwende einen kroatischen Muttersprachler (vorzugsweise Standard-Zagreber Dialekt)
    \item Aufnahme mit 44,1 kHz, 192 kbps MP3 oder höherer Qualität
    \item Deutlich und in moderatem Tempo für Lernende aussprechen
    \item Für Minimalpaare beide Wörter mit Pause dazwischen aufnehmen
    \item Für Sätze in normaler Sprechgeschwindigkeit aufnehmen
    \item 0,5 Sekunden Stille vor und nach jeder Aufnahme einfügen
\end{itemize}

\section{Schnellreferenztabelle}

\begin{center}
\begin{longtable}{lllll}
\caption{Kroatische Sonderbuchstaben - Schnellreferenz}\\
\toprule
\textbf{Buchstabe} & \textbf{IPA} & \textbf{Wie im Deutschen} & \textbf{Beispiel} & \textbf{Bedeutung} \\
\midrule
\endfirsthead
\multicolumn{5}{c}%
{\tablename\ \thetable\ -- Fortsetzung von vorheriger Seite} \\
\toprule
\textbf{Buchstabe} & \textbf{IPA} & \textbf{Wie im Deutschen} & \textbf{Beispiel} & \textbf{Bedeutung} \\
\midrule
\endhead
\midrule
\multicolumn{5}{r}{Fortsetzung auf nächster Seite...} \\
\endfoot
\bottomrule
\endlastfoot
Č, č & [tʃ] & tsch (Tschüss) & čokolada & Schokolade \\
Ć, ć & [tɕ] & weiches tsch & mačka & Katze \\
Š, š & [ʃ] & sch (Schule) & škola & Schule \\
Ž, ž & [ʒ] & stimmhaftes sch & žena & Frau \\
Dž, dž & [dʒ] & dsch (Dschungel) & džem & Marmelade \\
Đ, đ & [dʑ] & weiches dsch & đak & Schüler \\
C, c & [ts] & z (Zeit) & cura & Mädchen \\
Lj, lj & [ʎ] & palatalisiertes l & ljeto & Sommer \\
Nj, nj & [ɲ] & palatalisiertes n & konj & Pferd \\
R, r & [r] & gerolltes r & riba & Fisch \\
H, h & [x] & ch (Bach) & hrana & Essen \\
\end{longtable}
\end{center}

\section{Zusätzliche Ressourcen}

\textbf{Empfohlene Übung:}
\begin{itemize}
    \item Tägliche Ausspracheübung: 10-15 Minuten
    \item Konzentriere dich auf einen besonderen Laut pro Tag
    \item Verwende Audiomaterialien zusammen mit diesem Leitfaden
    \item Übe mit kroatischer Musik und Filmen
    \item Nimm dich selbst auf und vergleiche mit Muttersprachlern
\end{itemize}

\textbf{Online-Ressourcen:}
\begin{itemize}
    \item Forvo.com - Aussprache von Muttersprachlern
    \item YouTube kroatische Sprachkanäle
    \item Kroatisches Radio und TV (HRT)
    \item Sprachaustausch-Plattformen (Tandem, HelloTalk)
\end{itemize}

\vspace{1cm}

\begin{center}
\textbf{\Large Sretno s učenjem izgovora!}\\
\textbf{\Large Viel Erfolg beim Aussprachetraining!}
\end{center}
